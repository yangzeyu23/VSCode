\documentclass{ctexart}

\usepackage{yzy}

\title{光栅衍射实验报告}
\class{物理 32}
\name{杨泽宇}
\id{2023011329}

\begin{document}
\maketitle

\section*{摘要}
本实验通过调整与使用分光计,测定给定光栅的光栅常数与光波波长,从而进一步熟悉分光计,以及加深对光栅特性和光栅衍射公式的了解。

\section{实验原理}

理想光栅可视为许多平行等距等宽狭缝。对于光栅常数为$d$的光栅,设有一束波长为$\lambda$的平行光与之法线成角度$i$,入射于其上产生衍射,则衍射光谱中的第$m$级主极大衍射角$\varphi_m$满足:
\begin{equation}
  d(\sin\varphi_m\pm \sin i)=m\lambda
\end{equation}
其中$m$为衍射光谱的级次,取值为$0,\pm 1,\pm 2$等。当入射光线和衍射光线居于光栅法线同侧时,$(1.1)$式中等号
左边括号内取正号;两者分居法线异侧时则取负号。

当光线正入射时,角度$i=0$,此时上式变为:
\begin{equation}
  d\sin\varphi_m=m\lambda
\end{equation}
根据以上两个公式即可对相关物理量进行测量。

\begin{figure}[h]                                           
  \includegraphics[scale=0.18]{1.jpg}
  \qquad
  \includegraphics[scale=0.3]{2.jpg}
\end{figure}

对于$(1.1)$中对应同侧的情形,称$\Delta=\varphi_m+i$为偏向角,则易证$\Delta(i)$当且仅当$\varphi_m=i$(且居于法线同侧)时取得最小值,记作最小偏向角$\delta$,即有:
\begin{equation}
  i=\varphi_m=\displaystyle\frac{\delta}{2}, \quad 2d\sin\displaystyle\frac{\delta}{2}=m\lambda
\end{equation}
故亦可利用最小偏向角法对相关物理量进行测量。


\section{实验仪器及实验内容}

\subsection{仪器介绍}

\subsubsection*{1)分光计}
分光计主要由平行光管、自准直望远镜、刻度盘和载物平台构成。在本实验中,入射光由光源及平行光管提供,通过平台上的光栅发生衍射,由望远镜进行观察,并最终通过刻度盘完成角度测量。
\subsubsection*{2)光栅}
光栅上有许多平行等距的刻线。在本实验中,应使光栅刻痕与分光计铅锤主轴平行,否则谱线的倾斜将会影响测量结果。为调节方便,放置光栅时应使其平面垂直于小平台两个调水平螺钉的连线。
\subsubsection*{3)汞灯}
在可见光范围内,汞灯谱线中强度较高的有四条,波长分别为 435.8nm(紫),546.1nm(绿),577.0nm(黄)以及579.1nm(黄),本实验测量即将用到这些谱线。在使用过程中,汞灯必须与扼流圈串接,不能直接接 220 伏电源,否则会烧毁;
不能被频繁启闭,否则会降低使用寿命。
\subsection{实验步骤}
\subsubsection*{1)分光计调节}
\noindent \textbf{1.粗调:}凭眼睛判断,尽可能调节望远镜与平行光管光轴与刻度盘平行,调节小平台与刻度盘平行。\\
\noindent \textbf{2.调节望远镜:} 利用平面镜,调整目镜及调焦旋钮,使反射绿十字像清晰且与叉丝无视差;利用减半逐步逼近法,调节望远镜光轴垂直于分光计主轴。\\
\noindent \textbf{3.调节平行光管:}调节平行光管调焦旋钮,并调整狭缝宽度至0.5mm左右,使平行光管产生细锐清晰的平行光;调节至望远镜中狭缝像中点与叉丝中心交点重合,使平行光管垂直于分光计主轴。
\subsubsection*{2)正入射时的测定}
\noindent \textbf{1.调节光栅:}放置光栅使其平面与平行光管光轴垂直,调节载物盘螺钉使谱线等高。\\
\noindent \textbf{2.测定$\varphi_m$:}对同一波长的光,分别测量衍射零级两侧对应级次的衍射线的夹角$2\varphi_m$;注意同时对左右两侧游标读数,以消除刻度盘偏心差。\\
\noindent \textbf{3.测定$d$和$\lambda$:}先利用已知的汞灯绿线波长546.1nm(约定真值),由
测出的绿线衍射角求出光栅常数 $d$;再利用已求出的$d$与测得的衍射角求出汞灯的黄线波长(较大者),并计算不确定度。
\subsubsection*{3)斜入射时的测定}
\noindent \textbf{1.确定入射角:}调整光栅平面法线与平行光管光轴夹角为$15°$,并记下入射光方位与光栅平面法线方位。\\
\noindent \textbf{2.测定$\varphi_m$:}分别测出分居入射光两侧且属同一级次的谱线的衍射角,并判断其关于法线的方位。\\
\noindent \textbf{3.测定$\lambda$:}确定$m$的符号,并用已求出的$d$计算出汞灯光谱中黄线波长(较大者),并与约
定真值比较。
\subsubsection*{4)最小偏向角法的测定}
\noindent \textbf{1.确定$\delta$:}改变入射角,找到黄光某一条谱线与零级谱线的偏离为最小的方位,测出最小偏向角$\delta$。\\
\noindent \textbf{2.测定$\lambda$:}利用已求出的$d$和谱线级次对应的$\delta$,计算出黄线波长,并与约定真值比较。

\clearpage
\section{数据处理}

\subsection*{一.正入射测光栅常数及波长}
\noindent 入射光方位:$\varphi_{0\text{左}}=77°8',\quad \varphi_{0\text{右}}=257°5'$\\
\noindent $\Delta_{\text{仪}}=1'$


\begin{table}[!htbp]
  \centering
  \caption{正入射测定数据}\vspace{0.7em} \label{tab:aStrangeTable}%添加标题 设置标签
\begin{tabular}{c|ccc|ccc|cc}
  \toprule
  & $\varphi_{+1\text{左}}$ & $\varphi_{-1\text{左}}$ & $2\varphi_{1\text{左}}$ & $\varphi_{+1\text{右}}$ & $\varphi_{-1\text{右}}$ & $2\varphi_{1\text{右}}$ & $2\bar{\varphi_{1}}$  & $\bar{\varphi_{1}}$\\
  \midrule
  绿色 & $86°34'$ & $67°41'$ & $18°53'$ & $266°30'$ & $247°39'$ & $18°51'$ & $18°52'$ & $9°26'$ \\
  黄色 & $87°8'$  & $67°9'$  & $19°59'$ & $267°5'$  & $247°5'$  & $20°0'$  & $20°0'$  & $10°0'$ \\
  \bottomrule
  \end{tabular}
\end{table}
\noindent 由以上数据可作计算:
\begin{align}
\bar{d}&=\displaystyle\frac{\lambda_{\text{绿}}}{\sin{\bar{\varphi_{1\text{绿}}}}}=\displaystyle\frac{546.1\mathrm{nm}}{\sin{9°26'}}=3331.9\mathrm{nm}, \notag\\
U_d&=| \displaystyle\frac{\partial d}{\partial \varphi_m}| \Delta \varphi_m=\displaystyle\frac{\lambda_{\text{绿}}\cos{\bar{\varphi_{1\text{绿}}}}}{\sin^2{\bar{\varphi_{1\text{绿}}}}}\cdot \displaystyle\frac{\sqrt{2}}{2}\Delta_{\text{仪}}=\displaystyle\frac{546.1\mathrm{nm}\times \cos{9°26'}}{\sin^2{9°26'}}\times \displaystyle\frac{\sqrt{2}}{2}\times \displaystyle\frac{2\pi}{360\times 60}=4.1\mathrm{nm}. \notag
\end{align}
\textbf{故给定光栅的光栅常数为$\bar{d}\pm U_d=3332\pm 4 \mathrm{nm}$。}\\
与参考值$3333.3\mathrm{nm}$对比,偏差为$-0.039\%$。\\

\noindent 由此可计算黄光波长:
\begin{align}
\bar{\lambda}_{\text{黄}}&=\bar{d}\sin{\bar{\varphi_{1\text{黄}}}}=3332\mathrm{nm}\times \sin{10°0'}=578.6\mathrm{nm}, \notag\\
U_{\lambda}&=\sqrt{(\sin{\bar{\varphi_{1\text{黄}}}}U_d)^2+(\bar{d}\cos{\bar{\varphi_{1\text{黄}}}}\displaystyle\frac{\sqrt{2}}{2}\Delta_{\text{仪}})^2}=\sqrt{(\sin{10°0'}\times 4 \mathrm{nm})^2+(3332\mathrm{nm}\times\cos{10°0'}\times \displaystyle\frac{\sqrt{2}}{2}\displaystyle\frac{2\pi}{360\times 60})^2}=0.97\mathrm{nm}. \notag
\end{align}
\textbf{故(波长较大的)黄光波长为$\bar{\lambda}_{\text{黄}}\pm U_{\lambda}=578.6\pm 1.0 \mathrm{nm}$。}\\
与参考值$579.1\mathrm{nm}$对比,偏差为$-0.086\%$。

\subsection*{二.斜入射测波长}
\noindent 入射光方位:$\quad \varphi_{0\text{左}}=62°8',\quad \varphi_{0\text{右}}=242°5'$\\
\noindent 光栅法线方位:$\varphi_{0\text{左}}=77°8',\quad \varphi_{0\text{右}}=257°5'$ \quad \quad 入射角$i=15°0'$ \\
\noindent $\Delta_{\text{仪}}=1'$\\

\begin{table}[!htbp]
  \centering
  \caption{斜入射测定数据}\vspace{0.7em} \label{tab:aStrangeTable}%添加标题 设置标签
  \begin{tabular}{ccccc}
  \toprule
  & $\varphi_{1\text{左}}$ & $\varphi_{1\text{右}}$ & $2\varphi_1$ & $\bar{\varphi_1}$ \\
  \midrule
  $+1$级 & $51°30'$ & $231°28'$  & $51°15'$  & $25°38'$ \\
  $-1$级 & $72°17'$ & $252°13'$  & $9°44'$  & $4°52'$ \\
  \bottomrule
\end{tabular}
\end{table}

\noindent \textbf{由此可见,当$i=15°0'$时,$\pm 1$级的衍射光线均与入射光线分居于光栅平面法线异侧。}\\
\clearpage
\noindent 由$+1$级计算黄光波长:
\begin{align}
  \bar{\lambda}_{\text{黄}}&=\bar{d}(\sin{\bar{\varphi}_{+1\text{黄}}}-\sin{i})=3332\mathrm{nm}\times (\sin{25°38'}-\sin{15°0'})=579.07\mathrm{nm}, \notag\\
  U_{\lambda}&=\sqrt{[(\sin{\bar{\varphi}_{+1\text{黄}}}-\sin{i})U_d]^2+(\bar{d}\cos{\bar{\varphi}_{+1\text{黄}}}\displaystyle\frac{\sqrt{2}}{2}\Delta_{\text{仪}})^2}\notag \\
  &=\sqrt{[(\sin{25°38'}-\sin{15°0'})\times 4 \mathrm{nm}]^2+(3332\mathrm{nm}\times\cos{25°38'}\times \displaystyle\frac{\sqrt{2}}{2}\displaystyle\frac{2\pi}{360\times 60})^2}=0.93\mathrm{nm}. \notag
\end{align}
\noindent \textbf{故(波长较大的)黄光波长为$\bar{\lambda}_{\text{黄}}\pm U_{\lambda}=579.1\pm 1.0 \mathrm{nm}$。}\\
与参考值$579.1\mathrm{nm}$对比,中心值偏差几乎为0。\\

\noindent 由$-1$级计算黄光波长:
\begin{align}
  \bar{\lambda}_{\text{黄}}&=-\bar{d}(\sin{\bar{\varphi}_{-1\text{黄}}}-\sin{i})=-3332\mathrm{nm}\times (\sin{4°52'}-\sin{15°0'})=579.7\mathrm{nm}, \notag\\
  U_{\lambda}&=\sqrt{[(\sin{\bar{\varphi}_{-1\text{黄}}}-\sin{i})U_d]^2+(\bar{d}\cos{\bar{\varphi}_{-1\text{黄}}}\displaystyle\frac{\sqrt{2}}{2}\Delta_{\text{仪}})^2}\notag \\
  &=\sqrt{[(\sin{4°52'}-\sin{15°0'})\times 4 \mathrm{nm}]^2+(3332\mathrm{nm}\times\cos{4°52'}\times \displaystyle\frac{\sqrt{2}}{2}\displaystyle\frac{2\pi}{360\times 60})^2}=0.98\mathrm{nm}. \notag
\end{align}
\textbf{故(波长较大的)黄光波长为$\bar{\lambda}_{\text{黄}}\pm U_{\lambda}=579.7\pm 1.0 \mathrm{nm}$。}\\
与参考值$579.1\mathrm{nm}$对比,偏差为$+0.11\%$。

\subsection*{三.最小偏向角法测波长}

\noindent $\Delta_{\text{仪}}=1'$\\
\begin{table}[!htbp]
  \centering
  \caption{最小偏向角测定数据}\vspace{0.7em} \label{tab:aStrangeTable}%添加标题 设置标签
  \begin{tabular}{cccc}
  \toprule
  & 入射光方位 & $\delta$方位 & $\delta_{\text{左/右}}$\\
  \midrule
  $\varphi_{1\text{左}}$ & $72°22'$ & $82°19'$  & $9°57'$  \\
  $\varphi_{1\text{右}}$& $252°17'$ & $262°16'$ & $9°59'$ \\
  \bottomrule
\end{tabular}
\end{table}

\noindent 故$\bar{\delta}=9°58'$,由此可作计算:
\begin{align}
\bar{\lambda}_{\text{黄}}&=2\bar{d}\sin{\displaystyle\frac{\bar{\delta}}{2}}=2\times 3332\mathrm{nm}\times \sin{4°59'}=578.87\mathrm{nm}, \notag\\
U_{\lambda}&=\sqrt{(2\sin{\displaystyle\frac{\bar{\delta}}{2}}U_d)^2+(\bar{d}\cos\displaystyle\frac{{\bar{\delta}}}{2}\displaystyle\frac{\sqrt{2}}{2}\Delta_{\text{仪}})^2}\notag\\
&=\sqrt{(2\sin{4°59'}\times 4 \mathrm{nm})^2+(3332\mathrm{nm}\times\cos{4°59'}\times \displaystyle\frac{\sqrt{2}}{2}\displaystyle\frac{2\pi}{360\times 60})^2}=0.97\mathrm{nm}. \notag
\end{align}
\textbf{故(波长较大的)黄光波长为$\bar{\lambda}_{\text{黄}}\pm U_{\lambda}=578.9\pm 1.0 \mathrm{nm}$。}\\
与参考值$579.1\mathrm{nm}$对比,偏差为$-0.035\%$。
  
\section{结论}

在本次实验中,我们分别运用了正入射法、斜入射法和最小偏向角法,测定了给定光栅的光栅常数和汞灯的黄光波长,并加深了对光栅衍射现象的理解和对分光计相关操作的认识。

经过数据处理,我们得到的各结果与参考值的偏差均在$\pm0.2\%$以内,表明我们的测定较为精确。同时,\textbf{实验中也发现了以下问题:}
\begin{itemize}
  \item 望远镜视场内有杂光干扰,可能来自于室内普照灯或其他汞灯光源,一定程度上影响了谱线观察。
  \item 汞灯光谱中双黄线角距离很小,本应选择更高的衍射级次进行观测,但在本次实验过程中,$\pm2,3$等级次的谱线亮度和辨识度均较低,不适于测量,其原因有待进一步指明。
  \item 本台分光计刻度盘最大偏心差达到$6'$,一定程度上影响了测量准确度。
\end{itemize}

\noindent \textbf{针对这些问题,我们有以下建议:}\\
\noindent 1.可在汞灯周围适当增加遮光板,或在实验室外增加遮光帘,以减小外部光源对实验的影响。\\
\noindent 2.可对影响衍射光线亮度,对比度及清晰度的因素进行进一步探究。\\
\noindent 3.可考虑选用更高精度的实验仪器,以提高测量的准确度。\\
\noindent 4.另外,本实验探究的是透射式光栅的衍射现象,未涉及反射式光栅,可在后续实验中进行进一步探究。


\section{原始数据}

\begin{figure}[h]
  \centering                                          
  \includegraphics[scale=0.25]{1.png}
  \end{figure}
\end{document}