\documentclass{ctexart}

\usepackage{yzy}

\title{长训班力学试题解答}
\class{2025.1.15}
\name{鸟飞的学生们}


\begin{document}
\maketitle

\section*{第1题(40分)}

\begin{figure}[H]
  \centering
  \includegraphics[scale=0.6]{5.png}
\caption*{题解示意图,来自《奥赛物理题选》,注意数据不太一样}
\end{figure}
显然,$T$始终在$S_1,S_2$的垂直平分线$y$轴上运动. 记$t$时刻$P$的速度为$v$,坐标为$(x(t),y(t))$. 由几何关系有:
则由速度投影:
\begin{equation}

    \mathrm{d}l_1(t)=v(t)\mathrm{d}t \cos\theta_1(t),\quad \mathrm{d}l_2(t)=v(t)\mathrm{d}t \cos\theta_2(t)
    \tag{1.1}
\end{equation} 
作比并运用几何关系,得到:
\begin{equation}
    \frac{\mathrm{d}l_1(t)}{\mathrm{d}l_2(t)}=\frac{\cos\theta_1(t)}{\cos\theta_2(t)}=\frac{l_1(t)}{l_2(t)} 
    \tag{1.2}
\end{equation}
于是
\begin{equation}
  
    \frac{\mathrm{d}l_1(t)}{\mathrm{d}l_2(t)}=\frac{l_1(t)-\mathrm{d}l_1(t)}{l_2(t)-\mathrm{d}l_2(t)} =\frac{l_1(t)}{l_2(t)} 
    \tag{1.3}
\end{equation}
可见该比值恒定,而初值条件给出:
\begin{equation}
    \frac{l_1(t)}{l_2(t)}=\frac{l_1(0)}{l_2(0)}=\frac{\sqrt{50}}{\sqrt{48}} \tag{1.4}
\end{equation}
同时有几何关系:
\begin{equation}
    l_1(t)=\sqrt{(x+40)^2+y^2},\quad l_2(t)=\sqrt{(x-40)^2+y^2} \tag{1.5}
\end{equation}
即得:
\begin{equation}
    \frac{\sqrt{(x+40)^2+y^2}}{\sqrt{(x-40)^2+y^2}}=\frac{\sqrt{50}}{\sqrt{48}} \tag{1.6}
\end{equation}
化简得到:
\begin{equation}
    \Big(x+\frac{370}{3}\Big)^2+y^2=\Big(\frac{350}{3}\Big)^2 \tag{1.7}
\end{equation}
可见$P$的轨迹是圆心位于$\displaystyle \Big(-\frac{370}{3},0 \Big)$,半径为$\displaystyle\frac{350}{3}$的圆(单位已省略).

\noindent $P$的终点$(x_0,y_0)$对应于$T$处于(0,0)的位置,此时$\overline{PT}=40$,故附加条件为:
\begin{equation}
    x_0^2+y_0^2=40^2 \tag{1.8}
\end{equation}
联立(1.7)(1.8)可解得:
\begin{equation}
    x_0=-\frac{480}{37},\quad y_0=\frac{1400}{37} \tag{1.9}
\end{equation}
由此可确定$P$全过程转过的圆心角(可以利用向量点积):
\begin{equation}
    \alpha=17.9453° \tag{1.10}
\end{equation}
故所用的时间为:
\begin{equation}
    t=\frac{r\alpha}{v}=\frac{350/3\times 0.3132}{9}=4.06\mathrm{s} \tag{1.11}
\end{equation}
\\
\\

\section*{第2题(40分)}

\begin{figure}[H]
  \centering
  \includegraphics[scale=0.35]{6.png}
\caption*{坐标系示意}
\end{figure}
\noindent 建立如图所示的坐标系:
\begin{equation}
\vec{\varOmega}=\frac{\sqrt{3}}{2}\varOmega\hat{z}-\frac{1}{2}\varOmega\hat{x}\tag{2.1}
\end{equation}
近似下的动力学方程为:
	\begin{equation}
  \begin{cases}
		\displaystyle \ddot{z}=-\frac{GM}{(R+z)^2}\\
		\ddot{x}=\sqrt{3}\varOmega\dot{y}\\
		\ddot{y}=-\sqrt{3}\varOmega\dot{x}-\frac{1}{2}\varOmega\dot{z}
	\end{cases}
  \tag{2.2}
  \end{equation}
进一步近似,只保留一阶小量:
	\begin{equation}
  \begin{cases}
		\ddot{x}=0\\
		\ddot{y}=-\frac{1}{2}\varOmega\dot{z}
	\end{cases} 
  \tag{2.3}
  \end{equation}
	先来讨论$z$方向:
	\begin{equation}
  \frac{1}{2}\dot{z}^2=GM(\frac{1}{R+z}-\frac{1}{2R}) \tag{2.4}
  \end{equation}
	\begin{equation}
  \dot{z}=-\sqrt{\frac{GM(R-z)}{R(R+z)}} \tag{2.5}
  \end{equation}
	\begin{equation}
  t(z)=\int_z^R\sqrt{\frac{R(R-z)}{GM(R+z)}}\mathrm{d}z=\sqrt{\frac{R^3}{GM}}\left(\frac{\pi}{2}-\arcsin{\frac{z}{R}}+\sqrt{1-\left(\frac{z}{R}\right)^2}\right) \tag{2.6}
  \end{equation}
	故落到地面的时间为:
  \begin{equation}
  t_0=\sqrt{\frac{R^3}{GM}}\left(\frac{\pi}{2}+1\right) \tag{2.7}
  \end{equation}
	而在$y$方向上有:
	\begin{equation}
  \dot{y}=\frac{1}{2}\varOmega(2R-z) \tag{2.8}
  \end{equation}
	\begin{equation}
  y=\varOmega Rt_0-\frac{1}{2}\varOmega\int_0^{t_0}z(t)\mathrm{d}t \tag{2.9}
  \end{equation}
	分部积分:
	\begin{equation}
  \int_0^{t_0}z(t)\mathrm{d}t=tz(t)\Big|_0^{t_0}+\int_0^{R}t(z)\mathrm{d}z=\int_0^{R}t(z)\mathrm{d}z
  \tag{2.10}
  \end{equation}
  由此得到最终$y$方向移动的距离为:
	\begin{align*}
		y_f=\varOmega Rt_0-\frac{1}{2}\varOmega\int_0^{R}t(z)\mathrm{d}z&=\varOmega R\sqrt{\frac{R^3}{GM}}\Big[\left(\frac{\pi}{2}+1\right)-\frac{1}{2}\int_0^1\left(\frac{\pi}{2}-\arcsin{x}+\sqrt{1-x^2}\right)\mathrm{d}x\Big]\notag\\
		&=\left(\frac{1}{2}+\frac{3}{8}\pi\right)\varOmega R\sqrt{\frac{R}{g}}\approx627 \mathrm{km}\tag{2.11}
	\end{align*}
	故最终将落在原地点偏东627km处.

  来考察近似的误差影响,由量纲分析,$x$方向上的位移相对于$y$方向上的位移的比值量级大致为:
  \[\frac{x_f}{y_f}=\varOmega\sqrt{\frac{R}{g}}\approx0.05\]
  因此忽略是合理的,最终的结果可取为偏东$6.3\times 10^5\mathrm{m}$.

\clearpage
\section*{第3题(40分)}

先来分析本题的运动过程. 记小盘转动角速度为$\omega_1$,角加速度为$\beta_1$;相应地,环的转动角速度为$\omega_2$,角加速度为$\beta_2$,质心速度为$v$,质心\textbf{朝左}的平动加速度为$a$,小盘的半径为$R$,质量为$m$. 

一开始时,圆环因受地面朝左的滑动摩擦力,将会作顺时针转动,同时通过对小盘的摩擦力带动小盘也作顺时针转动。此时若$\mu$足够大,两者可一起转动而无相对滑动. 而当$\mu$较小时,小盘的转动会落后于环的转动,出现相对滑动。以下先讨论会出现相对滑动的情况.

\noindent \textbf{第一阶段:}

当$\beta_1<\beta_2,\omega_1<\omega_2$时,环和盘之间存在相对滑动,$a$对$v$起减速作用.当达到纯滚条件在$\omega_2\cdot 2R=v$时,地面对环的向左滑动摩擦力消除.

\noindent \textbf{第二阶段:}

因小盘转速小于环的转速$\omega_2$,两者间滑动摩擦继续存在,使$\omega_1$继续加速,$\omega_2$继续减速,环与地接触点又有相对滑动趋势,环又会受到地面摩擦力. 

此时,环与地摩擦力不能朝右,否则质心会加速,环角速度$\omega_2$会减小,使接触点朝右运动,这显然与摩擦力朝右矛盾. 因此这一摩擦力只能朝左,但又不能是滑动摩擦力;否则动力学方程与第一阶段完全一致,将使$\omega_2$继续加速,质心继续减速,环的触地点左行,这与摩擦力朝左矛盾.
综上所述,这一阶段\textbf{环与地摩擦力只能是朝左的静摩擦力},触地点仍为瞬心.这样,$\omega_2$减小,质心继续减速,且纯滚条件$\omega_2\cdot 2R=v, \beta_2\cdot 2R=a$始终成立. 当$\omega_2$减小到与$\omega_1$相等,小盘与环间无相对滑动时,系统即达到最终的稳定状态,所有摩擦力消失.

现在来分析小盘与环间的作用力. \textbf{注意盘与环的接触点不一定在盘的正下方,因而(法向)支持力不一定竖直向上.}但全过程中无论小盘与环间的力作用点在何位置,所受法向力$\bm{N_1}$与摩擦力$\bm{f_1}$间均存在关系:
\begin{equation}
    f_1=\mu N_1,\quad \bm{f_1}\bot \bm{N_1}\notag
\end{equation}
其合力$F_1=\sqrt{N_1^2+f_1^2}=\sqrt{1+\mu^2}N_1$,且考虑到Newton第二定律显然有$F_1=m\sqrt{g^2+a^2}$.这样即可解得:
\begin{equation}
    N_1=\frac{m\sqrt{g^2+a^2}}{\sqrt{1+\mu^2}},\quad f_1=\frac{\mu m\sqrt{g^2+a^2}}{\sqrt{1+\mu^2}}\notag
\end{equation}
注意,我们忽略全过程分段讨论时不同过程之间极短暂过渡过程的影响(事实上也无伤大雅).以下即来做具体计算.
\begin{figure}[H]
  \centering
  \includegraphics[scale=0.6]{1.png}
\end{figure}

\noindent 上图给出了第一阶段的受力分析. 对小盘有:
\begin{equation}
    F_{1\bot}=mg \tag{3.1}
\end{equation}
\begin{equation}
    F_{1\parallel}=ma \tag{3.2}
\end{equation}
\begin{equation}
    f_1=\mu m\sqrt{g^2+a^2}\Big/\sqrt{1+\mu^2} \tag{3.3}
\end{equation}
\begin{equation}
    f_1R=I_1\beta_1 ,\quad I_1=\frac{1}{2}mR^2 \tag{3.4}
\end{equation}
对环有:
\begin{equation}
    f_2-F_{1\parallel}=3ma \tag{3.5}
\end{equation}
\begin{equation}
    N_2=3mg+F_{1\bot} \tag{3.6}
\end{equation}
\begin{equation}
    f_2=\mu_0 N_2 \tag{3.7}
\end{equation}
\begin{equation}
    f_2\cdot 2R-f_1R=I_2\beta_2 ,\quad I_2=\frac{1}{2}3m[R^2+(2R)^2]=\frac{15}{2}mR^2 \tag{3.8}
\end{equation}
联立以上数式,可以解得:
\begin{equation}
    a=\mu_0 g \tag{3.9}
\end{equation}
\begin{equation}
    f_1=\sqrt{\frac{1+\mu_0^2}{1+\mu^2}}\mu mg \tag{3.10}
\end{equation}
\begin{equation}
\beta_1=2\mu\sqrt{\frac{1+\mu_0^2}{1+\mu^2}}g/R \tag{3.11}
\end{equation}
\begin{equation}
\beta_2=\frac{2}{15}\Big(8\mu_0-\mu \sqrt{\frac{1+\mu_0^2}{1+\mu^2}}\Big)g/R \tag{3.12}
\end{equation}
第一阶段的存在要求$\beta_1<\beta_2$,代入参数值$\mu_0=0.1$的即可解得$\mu$此时的范围为:
\begin{equation}
    0<\mu<0.0498 \tag{3.13}
\end{equation}
题给数据中$\mu=0.03$,故第一阶段存在,以上讨论成立. 具体代入数值,解得:
\begin{equation}
    \beta_1=0.06027\frac{g}{R},\quad \beta_2=0.1026\frac{g}{R} \tag{3.14}
\end{equation}
因此在第一阶段中,质心速度关于时间的变化为:
\begin{equation}
    v(t)=v_0-\mu_0 g t \tag{3.15}
\end{equation}
第一阶段结束的时刻$\tau$由纯滚条件给出:
\begin{equation}
    \beta_2 \tau\cdot 2R=v_0-\mu_0 g\tau \tag{3.16}
\end{equation}
解得:
\begin{equation}
    \tau=3.2765\frac{v_0}{g},\quad v(\tau)=0.6723v_0,\quad \omega_{1\tau}=0.1975\frac{v_0}{R},\quad \omega_{2\tau}=0.3362\frac{v_0}{R} \tag{3.17}
\end{equation}
然后进入第二阶段. 此时对于盘的分析都仍然成立,但是对于环而言,与地摩擦力和支持力的关系不再成立(此时是纯滚),应该代之以纯滚条件:
\begin{equation}
    a=\beta_2\cdot 2R \tag{3.18}
\end{equation}
此时环的角加速度$\beta_2$是逆时针正向的了,方向已经改变.故对环的转动方程相应应该改为:
\begin{equation}
    f_2\cdot 2R-f_1R=I_2\cdot (-\beta_2) \tag{3.19}
\end{equation}
为了表达的方便,引入$\mu_0'$使其满足:
\begin{equation}
f_2=\mu_0'N_2 \tag{3.20}
\end{equation}
这样,上述新得到的两个式子在数学形式上与之前的相同. 由此重新求解方程组,可得:
\begin{equation}
    a=\mu_0' g \tag{3.21}
\end{equation}
\begin{equation}
  \beta_1=2\mu\sqrt{\frac{1+\mu_0'^2}{1+\mu^2}}g/R \tag{3.22}
  \end{equation}
  \begin{equation}
  \beta_2=-\frac{2}{15}\Big(8\mu_0'-\mu \sqrt{\frac{1+\mu_0'^2}{1+\mu^2}}\Big)g/R \tag{3.23}
  \end{equation}
联立$a,\beta_2,\mu_0'$几式的关系,计算可得:
\begin{equation}
    \mu_0'=2.55\times 10^{-3},\quad a=2.55\times 10^{-3}g \tag{3.24}
\end{equation}
\begin{equation}
    \beta_1'=0.05997\frac{g}{R},\quad \beta_2'=0.00128\frac{g}{R}\quad \text{(注意正方向不同)} \tag{3.25}
\end{equation}
第二阶段从开始到结束的时间间隔$T$由终末角速度$\omega_1=\omega_2$给出:
\begin{equation}
    \omega_{1\tau}+\beta_1' T=\omega_{2\tau}-\beta_2' T \tag{3.26}
\end{equation}
解得:
\begin{equation}
    T=2.2185\frac{v_0}{g},\quad v(\tau+T)=v(\tau)-aT=0.6667v_0 \tag{3.27}
\end{equation}
故最终的图像可作出为:

\begin{figure}[H]
  \centering
  \includegraphics[scale=0.6]{2.png}
\caption*{质心速度关于时间的变化(速度以$v_0$为单位值,时间以$v_0/g$为单位值)其中蓝色为第一段,橙色为第二段}
\end{figure}

注意到质心的末态速度为$\frac{2}{3}v_0$,这是一个与摩擦系数值无关的结果. 事实上,对在地面参考系的地面上任意一点,系统的角动量守恒,故由初态和末态得:
\begin{equation}
4m \cdot v_0\cdot 2R=4mv_{\text{末}}\cdot 2R+(I_1+I_2)\omega_{2\text{末}},\quad \omega_{2\text{末}}=\frac{v_{\text{末}}}{2R} \tag{3.28}
\end{equation}
即可解得$v_{\text{末}}=\frac{2}{3}v_0$.

\clearpage
\section*{第4题(50分)}
\begin{figure}[htbp]
  \centering
  \subfloat[第一问]
  {\includegraphics[width=0.4\textwidth]{3.png}}
  \quad    
  \subfloat[第二问]
  {\includegraphics[width=0.35\textwidth]{4.png}}
  \caption*{任泓锦同学手绘的两小问图解}
\end{figure}

\noindent(1)
\noindent 待定的冲量分布如图所示. 首先来分析A点的运动,由动量定理得:
\begin{equation}
   m'v _{Ax}=I_A-I_1,\quad v_{Ay}=0,\qquad m'\equiv \frac{1}{6}m \tag{4.1}
\end{equation}
A,C两点沿AC杆速度相同给出:
\begin{equation}
    \frac{(I_A-I_1)/2}{m'}=\frac{I_1-I_2/2}{m'} \quad \Rightarrow \quad  I_A=3I_1-I_2 \tag{4.2}
\end{equation}
同样地,对CD有:
\begin{equation}
    \frac{1}{2}I_1-I_2=I_2-\frac{1}{2}I_3 \quad \Rightarrow \quad  I_1-4I_2+I_3=0  \tag{4.3}
\end{equation}
对BD有:
\begin{equation}
    \frac{1}{2}I_3=\frac{1}{2}I_2-I_3 \quad \Rightarrow \quad  3I_3=I_2 \tag{4.4}
\end{equation}
对B点有:
\begin{equation}
    I_3=m'v_B \tag{4.5}
\end{equation}
联立以上数式,可以解得:
\begin{equation}
   I_1=11I_3,\quad I_3=\frac{1}{30}mv,\quad v_B=\frac{1}{5}v  \tag{4.6}
\end{equation}

\noindent(2)类似地,我们先设出沿绳方向的冲量的角分布$I(\theta)$. 注意是软绳,故绳上各点的冲量只能沿绳方向。绳上各点的径向速度与切向速度用$v_r,v_\tau$描述。

由动量定理:
\begin{equation}
v_r(\theta)=-\frac{2\pi}{m}I(\theta),v_\tau(\theta)=\frac{2\pi}{m}\frac{\mathrm{d}I(\theta)}{\mathrm{d}\theta} \tag{4.7}
\end{equation}
此外,考虑绳不可伸长的约束条件,取绳的自然坐标系
\begin{equation}
\hat{\tau}\cdot\frac{\partial\bm{v}}{\partial{s}}=0 \tag{4.8}
\end{equation}
其中$\hat{\tau}$为切向单位矢量,$s$为绳的弧长参数. 展开至径向与切向,有:
\begin{equation}
  \frac{\partial\bm{v}}{\partial{s}}=\frac{\partial v_r}{\partial s}\hat{r}+\frac{\partial v_\tau}{\partial s}\hat{\tau}+v_r\frac{\partial \hat{r}}{\partial s}+v_{\tau}\frac{\partial \hat{\tau}}{\partial s} \tag{4.9}
\end{equation}
注意到:
\begin{equation}
  \frac{\partial \hat{r}}{\partial s}=\frac{\hat{\tau}}{\rho},\quad \frac{\partial \hat{\tau}}{\partial s}=-\frac{\hat{r}}{\rho} \tag{4.10}
\end{equation}
其中$\rho$为绳的曲率半径,因此绳的不可伸长条件即为:
\begin{equation}
  \frac{\partial v_{\tau}}{\partial s}+\frac{v_r}{\rho}=0 \tag{4.11}
\end{equation}
回到这个问题,我们知道$\rho=R,\mathrm{d}s=R\mathrm{d}\theta$,故上式化为:
\begin{equation}
  \frac{\mathrm{d} v_\tau}{\mathrm{d} \theta}+{v_r}=0 \tag{4.12}
\end{equation}
代入$v_r,v_\tau$的表达式,可得:
\begin{equation}
  \frac{\mathrm{d}^2 I(\theta)}{\mathrm{d} \theta^2}-{I(\theta)}=0 \tag{4.13}
\end{equation}
这个方程的通解为:
\begin{equation}
  I(\theta)=C_1\mathrm{e}^{\theta}+C_2\mathrm{e}^{-\theta},\quad \theta\in[0,\pi] \tag{4.14}
\end{equation}
圆环上下的对称性要求$I(\theta)=I(2\pi-\theta)$,然后来考虑如何通过边界条件来定出$C_1,C_2$。先来看$B$点,此点不受外冲量,故$B$点处$\frac{\mathrm{d} I(\theta)}{\mathrm{d} \theta}$连续:
\begin{equation}
  \frac{\mathrm{d} I(\theta)}{\mathrm{d} \theta}\Big|_{\theta=\pi^-}=-\frac{\mathrm{d} I(\theta)}{\mathrm{d} \theta}\Big|_{\theta=\pi^+} \tag{4.15}
\end{equation}
由于
\begin{equation}
  \frac{\mathrm{d} I(\theta)}{\mathrm{d} \theta}\Big|_{\theta=\pi^+}=\frac{\mathrm{d} I(2\pi-\theta)}{\mathrm{d} \theta}\Big|_{\theta=\pi^+}=-\frac{\mathrm{d} I(\theta)}{\mathrm{d} \theta}\Big|_{\theta=\pi^-} \tag{4.16}
\end{equation}
联立以上两式,故$\frac{\mathrm{d} I(\theta)}{\mathrm{d} \theta}\Big|_{\theta=\pi^-}=\frac{\mathrm{d} I(\theta)}{\mathrm{d} \theta}\Big|_{\theta=\pi^+}=0$,即
\begin{equation}
  C_1\mathrm{e}^{\pi}-C_2\mathrm{e}^{-\pi}=0 \tag{4.17}
\end{equation}
再来看$A$点,此点受到的冲量为$I_A$,因此方程(4.13)实际上是不完整的。由于
\begin{equation}
  v_r(\theta)=\frac{2\pi}{m}(I_A\delta(\theta)-I(\theta)),\quad v_\tau(\theta)=\frac{2\pi}{m}\frac{\mathrm{d}I(\theta)}{\mathrm{d}\theta} \tag{4.18}
\end{equation}
其中$\delta(\theta)$为Dirac函数. 应用速度条件(4.12),完整的方程应写为:
\begin{equation}
  \frac{\mathrm{d}^2 I(\theta)}{\mathrm{d} \theta^2}-{I(\theta)}+I_A\delta(\theta)=0 \tag{4.19}
\end{equation}
将其对经过$\theta=0$处积分,得:
\begin{equation}
  \frac{\mathrm{d} I(\theta)}{\mathrm{d} \theta}\Big|_{\theta=0^+}-\frac{\mathrm{d} I(\theta)}{\mathrm{d} \theta}\Big|_{\theta=0^-}=-I_A \tag{4.20}
\end{equation}
由对称性仍有:
\begin{equation}
\frac{\mathrm{d} I(\theta)}{\mathrm{d} \theta}\Big|_{\theta=0^+}=-\frac{\mathrm{d} I(\theta)}{\mathrm{d} \theta}\Big|_{\theta=0^-} \tag{4.21}
\end{equation}
联立以上两式,可得:
\begin{equation}
  \frac{\mathrm{d} I(\theta)}{\mathrm{d}\theta}\Big|_{\theta=0}=C_1-C_2=-\frac{I_A}{2} \tag{4.22}
\end{equation}
联立(4.17)式与(4.22)式,可得:
\begin{equation}
  C_1=\frac{I_A}{2(\mathrm{e}^{2\pi}-1)},\quad C_2=-\frac{I_A\mathrm{e}^{2\pi}}{2(\mathrm{e}^{2\pi}-1)} \tag{4.23}
\end{equation}
故$I(\theta)$的具体形式为:
\begin{equation}
  I(\theta)=\frac{I_A}{2(\mathrm{e}^{2\pi}-1)}\Big(\mathrm{e}^{\theta}+\mathrm{e}^{2\pi-\theta}\Big) \tag{4.24}
\end{equation}
最后,$B$点的速度为:
\begin{equation}
  v_B=\frac{2\pi}{m}I(\pi)=\frac{\pi}{\sinh{\pi}}v \tag{4.25}
\end{equation}
\clearpage
\section*{第5题(50分)}
\noindent 本题首先需要默认如下假设:\textbf{小球是“超球”,与筒壁碰撞时接触点无相对运动,但是碰撞前后可以有相对运动.} 具体描述可参考下图.\par
    \begin{figure}[htbp]
        \centering
        \includegraphics[width=0.8\linewidth]{chaoqiu}
        \caption*{第38届复赛第三题}
        \label{chaoqiu}
    \end{figure}
    首先,使用对接触点(质心+相对质心)的角动量守恒方程和能量守恒方程(科尼希定理)计算一次普遍的碰撞的结果:\par
    \begin{equation}
        \left\{
        \begin{aligned}
            &mv_{1\parallel}r+\frac{2}{5}mr^2 \omega_1=mv_{0\parallel}r+\frac{2}{5}mr^2 \omega_0\\
            &\frac{1}{2}mv_{1\parallel}^2+\frac{1}{2} \Big(\frac{2}{5}\Big)mr^2 \omega_1^2=\frac{1}{2}mv_{0\parallel}^2+\frac{1}{2} \Big(\frac{2}{5}\Big)mr^2 \omega_0^2
        \end{aligned}
        \right.
        \Rightarrow 
        \left\{
        \begin{aligned}
            &v_{1\perp}=v_{0\perp}\\
            &v_{1\parallel}=\frac{3}{7}v_{0\parallel}+\frac{4}{7}\omega_0r\\
            &\omega_{1}r=\frac{10}{7}v_{0\parallel}+\frac{-3}{7}\omega_0r
        \end{aligned}
        \right. \tag{5.1}
    \end{equation}
    0表示初态,1表示第一次碰撞后的态,正方向自己领悟,求解过程中舍去了0和1相同的解。\par
    采用矩阵记号表示这一结果:\par
    \begin{equation}
        \begin{bmatrix}
            v_{1\parallel} \\
            \omega_1r
        \end{bmatrix}
        =\frac{1}{7}
        \begin{bmatrix}
            3 & 4 \\
            10 & -3 
        \end{bmatrix}
        \begin{bmatrix}
            v_{0\parallel} \\
            \omega_0r
        \end{bmatrix}\tag{5.2}
    \end{equation}
    小球除了与筒壁碰撞,其他时候作匀速直线运动,运动轨迹如下:\par
    \begin{figure}[htbp]
        \centering
        \includegraphics[width=0.3\linewidth]{yundong}
        \caption*{任泓锦同学手绘的小球运动轨迹}
        \label{yundong}
    \end{figure}
    容易知道:
    \begin{equation}
        \begin{bmatrix}
            v_{n\parallel} \\
            \omega_nr
        \end{bmatrix}
        =\frac{1}{7^n}
        \begin{bmatrix}
            3 & 4 \\
            10 & -3 
        \end{bmatrix}^n
        \begin{bmatrix}
            v_{0\parallel} \\
            \omega_0r
        \end{bmatrix}
        =
        \left\{
            \begin{aligned}
                &\frac{1}{7}
                \begin{bmatrix}
                    3 & 4 \\
                    10 & -3 
                \end{bmatrix}
                \begin{bmatrix}
                    v_{0\parallel} \\
                    \omega_0r
                \end{bmatrix}&(\text{n为奇数}) \\
                &\begin{bmatrix}
                    v_{0\parallel} \\
                    \omega_0r
                \end{bmatrix}&(\text{n为偶数})
            \end{aligned}
        \right . \tag{5.3}
    \end{equation}
事实上,每一次普遍碰撞满足的(二次)方程组都一样,给出不同的两个解,而每一次碰撞都会改变运动状态,故小球的运动参数依据奇偶性在两个解中来回切换.

\noindent     (1)此时$\omega_0=0,v_{0\parallel}=v_0\sin\theta_0$,由以上推导,$v_{n\perp}=v_{0\perp}=v_0\cos\theta_0,\theta_n=\arctan\frac{v_{n\parallel}}{v_{n\perp}}$
    ,故:
    \begin{equation}
        \theta_n=
        \left\{
            \begin{aligned}
                &\theta_1=\arctan(\frac{3}{7}\tan\theta_0)&(\text{n为奇数}) \\
                &\theta_0&(\text{n为偶数})
            \end{aligned}
        \right.\tag{5.4}
    \end{equation}
    \begin{equation}
        r_{\min}=(R-r)\sin\theta_1=(R-r)\frac{\frac{3}{7}\tan\theta_0}{\sqrt{1+\frac{9}{49}\tan^2\theta_0}} \tag{5.5}
    \end{equation}
    \noindent     (2)需要更换参考系使得外力不做功,便于列出能量守恒方程. 这里选择\textbf{瞬时碰撞点的平动参考系}(惯性力有限不予考虑),容易发现此时与前面的情形等价,则:
    \begin{equation}
        \begin{bmatrix}
            v_{n\parallel}-\varOmega R \\
            \omega_nr
        \end{bmatrix}
        =
        \left\{
            \begin{aligned}
                &\frac{1}{7}
                \begin{bmatrix}
                    3 & 4 \\
                    10 & -3 
                \end{bmatrix}
                \begin{bmatrix}
                    v_{0\parallel}-\varOmega R \\
                    \omega_0r
                \end{bmatrix}&(\text{n为奇数}) \\
                &\begin{bmatrix}
                    v_{0\parallel}-\varOmega R \\
                    \omega_0r
                \end{bmatrix}&(\text{n为偶数})
            \end{aligned}
        \right. \tag{5.6}
    \end{equation}
    此时$\omega_0=0,v_{0\parallel}=v_0\sin\theta_0,v_{n\perp}=v_{0\perp}=v_0\cos\theta_0$,即
    \begin{equation}
        \begin{bmatrix}
            v_{n\parallel}\\
            \omega_nr
        \end{bmatrix}
        =
        \left\{
            \begin{aligned}
                &\begin{bmatrix}
                    \frac{3}{7}v_0\sin\theta_0+\frac{4}{7}\varOmega R \\
                    \frac{10}{7}(v_0\sin\theta_0-\varOmega R)
                \end{bmatrix}&(\text{n为奇数}) \\
                &\begin{bmatrix}
                    v_0\sin\theta_0\\
                    0
                \end{bmatrix}&(\text{n为偶数})
            \end{aligned}
        \right. \tag{5.7}
    \end{equation}
    故
    \begin{equation}
        \theta_n=
        \left\{
            \begin{aligned}
                &\theta_1=\arctan(\frac{3}{7}\tan\theta_0+\frac{4\varOmega R}{7v_0\cos\theta_0})&(\text{n为奇数}) \\
                &\theta_0&(\text{n为偶数})
            \end{aligned}
        \right. \tag{5.8}
    \end{equation}
    \begin{equation}
        r_{\min}=(R-r)\min\{\sin\theta_1,\sin\theta_0\} \tag{5.9}
    \end{equation}
    \noindent     (3)
    换转动系,引入离心力和科里奥利力可得到牛顿运动方程:
    \begin{equation}
        \left\{
        \begin{aligned}
            &m\ddot{x}=m\varOmega^2 x+2m\varOmega \dot{y}\\
            &m\ddot{y}=m\varOmega^2 y-2m\varOmega \dot{x}
        \end{aligned}
        \right. \tag{5.10}
    \end{equation}
这是一个二阶线性方程组,解这类方程最简便的方法是复数法. 考虑设$\tilde{r}=x+iy$,我们可以把两个方程合成一个
\begin{equation}
  \ddot{\tilde{r}}+2i\varOmega \dot{\tilde{r}}-\varOmega^2 \tilde{r}=0 \tag{5.11}
\end{equation}
其判别式$\Delta=-4\varOmega^2+4\varOmega^2=0$,特征根为$\lambda=-i\varOmega$,故通解为:
\begin{equation}
  \tilde{r}=(A+Bt)\mathrm{e}^{-i\varOmega t} \tag{5.12}
\end{equation}
其中$A,B$为待定常数. 在转动系中,小球的初始条件为:
\begin{equation}
  \tilde{r}_i=-i(R-r) \tag{5.13}
\end{equation}
\begin{equation}
  \tilde{v}_i=v_0\sin{\theta_0}-\varOmega(R-r)+iv_0\cos{\theta_0} \tag{5.14}
\end{equation}
解得:
\begin{equation}
\begin{cases}
  A=-\varOmega(R-r)+v_0\sin{\theta_0}+iv_0\cos{\theta_0}\\
  B=-i(R-r)
\end{cases} \tag{5.15}
\end{equation}
故
\begin{equation}
\tilde{r}(t)=\left[\left(-\varOmega(R-r)+v_0\sin{\theta_0}+iv_0\cos{\theta_0}\right)t-i(R-r)\right]\mathrm{e}^{-i\varOmega t} \tag{5.16}
\end{equation}
实虚部分离得到:
\begin{equation}
  \begin{cases}
    x(t)=(-\varOmega(R-r)+v_0\sin{\theta_0})t\cos{\varOmega t}+(v_0t\cos{\theta_0}-(R-r))\sin{\varOmega t}\\
    y(t)=(\varOmega(R-r)-v_0\sin{\theta_0})t\sin{\varOmega t}+(R-r-v_0t\cos{\theta_0})\cos{\varOmega t}
  \end{cases} \tag{5.17}
  \end{equation}

这个式子是用参数方程表示的轨迹,虽然不是用显函数$y(x)$表达的,但学长已经拼尽全力无法战胜,还是将就一下吧.

\clearpage
\section*{第6题(50分)}
\noindent (1)先确定各轴承的支持力. 由竖直方向平衡:
\begin{equation}
  N_1+N_2+N_3=Mg \tag{6.1}
\end{equation}
由力矩平衡:
\begin{equation}
  N_1(l-y)=(N_2+N_3)\Big(\frac{1}{2}l+y\Big) \tag{6.2}
\end{equation}
\begin{equation}
  N_1x+N_2\Big(\frac{\sqrt{3}}{2}l+x\Big)=N_3\Big(\frac{\sqrt{3}}{2}l-x\Big) \tag{6.3}
\end{equation}
由于题目要求保留到一阶小,不妨直接设出支持力的形式为:
\begin{equation}
  N_1=\frac{1}{3}Mg\Big(1+\alpha_1\frac{x}{l}+\beta_1\frac{y}{l}\Big),\quad N_2=\frac{1}{3}Mg\Big(1+\alpha_2\frac{x}{l}+\beta_2\frac{y}{l}\Big),\quad N_3=\frac{1}{3}Mg\Big(1+\alpha_3\frac{x}{l}+\beta_3\frac{y}{l}\Big) \tag{6.4}
\end{equation}
代入(6.1)-(6.3)式,可得方程组:
\begin{equation}
  \begin{cases}
    \alpha_1+\alpha_2+\alpha_3=0 \notag\\
    3+\frac{\sqrt{3}}{2}\alpha_2-\frac{\sqrt{3}}{2}\alpha_3=0 \notag\\
    \alpha_1-\frac{1}{2}\alpha_2-\frac{1}{2}\alpha_3=0 \notag\\
    \beta_1+\beta_2+\beta_3=0 \notag\\
    \beta_1-\frac{1}{2}\beta_2-\frac{1}{2}\beta_3-3=0 \notag\\
    \frac{\sqrt{3}}{2}\beta_2-\frac{\sqrt{3}}{2}\beta_3=0 \notag
  \end{cases} \tag{6.5}
\end{equation}
解得:
\begin{equation}
  \alpha_1=0,\quad \alpha_2=-\sqrt{3},\quad \alpha_3=\sqrt{3},\quad \beta_1=2,\quad \beta_2=-1,\quad \beta_3=-1 \tag{6.6}
\end{equation}
再来分析个轴承与盘接触点的摩擦力. 计算可知圆盘在各接触点的相对速度矢量分别为:
\begin{equation}
  \bm{v}_1=(\dot{x},\dot{y}+v_0 ),\quad \bm{v_2}=\Big( \dot{x}-\frac{\sqrt{3}}{2}v_0,\dot{y}-\frac{1}{2}v_0\Big),\quad 
  \bm{v_3}=\Big( \dot{x}+\frac{\sqrt{3}}{2}v_0,\dot{y}-\frac{1}{2}v_0\Big) \tag{6.7}
\end{equation}
需要考虑摩擦力沿$x,y$轴的分量,故考虑如下精确到一阶的展开:
\begin{equation}
  \begin{cases}
\displaystyle \frac{\dot{x}}{\sqrt{\dot{x}^2+(\dot{y}+v_0)^2}}\sim \frac{\dot{x}}{v_0},\quad \frac{\dot{y}+v_0}{\sqrt{\dot{x}^2+(\dot{y}+v_0)^2}}\sim 1\\
\displaystyle\frac{\dot{x}-\frac{\sqrt{3}}{2}v_0}{\sqrt{(\dot{x}-\frac{\sqrt{3}}{2}v_0)^2+(\dot{y}-\frac{1}{2}v_0)^2}}\sim 
-\frac{\sqrt{3}}{2}+\frac{1}{4}\frac{\dot{x}}{v_0}-\frac{\sqrt{3}}{4}\frac{\dot{y}}{v_0},\quad \frac{\dot{y}-\frac{1}{2}v_0}{\sqrt{(\dot{x}-\frac{\sqrt{3}}{2}v_0)^2+(\dot{y}-\frac{1}{2}v_0)^2}}\sim -\frac{1}{2}-\frac{\sqrt{3}}{4}\frac{\dot{x}}{v_0}+\frac{3}{4}\frac{\dot{y}}{v_0}\\
\displaystyle\quad \frac{\dot{x}+\frac{\sqrt{3}}{2}v_0}{\sqrt{(\dot{x}+\frac{\sqrt{3}}{2}v_0)^2+(\dot{y}-\frac{1}{2}v_0)^2}}\sim 
\frac{\sqrt{3}}{2}+\frac{1}{4}\frac{\dot{x}}{v_0}+\frac{\sqrt{3}}{4}\frac{\dot{y}}{v_0},\quad \frac{\dot{y}-\frac{1}{2}v_0}{\sqrt{(\dot{x}+\frac{\sqrt{3}}{2}v_0)^2+(\dot{y}-\frac{1}{2}v_0)^2}}\sim -\frac{1}{2}+\frac{\sqrt{3}}{4}\frac{\dot{x}}{v_0}+\frac{3}{4}\frac{\dot{y}}{v_0}
  \end{cases}\notag
\end{equation}
由此可计算圆盘受到的合力:
\begin{align}
  \textbf{$x$方向:}  f_x=-\frac{1}{3}\mu Mg\Big[\frac{\dot{x}}{v_0}\cdot\Big(1&+2\frac{y}{l}\Big)+\Big(-\frac{\sqrt{3}}{2}+\frac{1}{4}\frac{\dot{x}}{v_0}-\frac{\sqrt{3}}{4}\frac{\dot{y}}{v_0}\Big)\cdot (1-\sqrt{3}\frac{x}{l}-\frac{y}{l})\notag\\
  &+\Big(\frac{\sqrt{3}}{2}+\frac{1}{4}\frac{\dot{x}}{v_0}+\frac{\sqrt{3}}{4}\frac{\dot{y}}{v_0}\Big)\cdot (1+\sqrt{3}\frac{x}{l}-\frac{y}{l})\Big] =-\mu Mg\Big(\frac{x}{l}+\frac{1}{2}\frac{\dot{x}}{v_0}\Big) \tag{6.8}
\end{align}
同样地,计算可得:
\begin{equation}
  \textbf{$y$方向:} f_y=-\mu Mg\Big(\frac{y}{l}+\frac{1}{2}\frac{\dot{y}}{v_0}\Big) \tag{6.9}
\end{equation}
\noindent (2)由于圆盘在$x,y$两个方向的受力数学形式相同,讨论$x$方向的运动情况即可. 由牛顿第二定律:
\begin{equation}
  \ddot{x}+\frac{\mu g}{2v_0}\dot{x}+\frac{\mu g}{l}x=0 \tag{6.10}
\end{equation}
令$\displaystyle\beta=\frac{\mu g}{4v_0},\omega_0=\sqrt{\frac{\mu g}{l}}$,则上式可化为$\ddot{x}+2\beta\dot{x}+\omega_0^2x=0$,以下来根据$\beta$与$\omega_0$的大小关系来讨论速度满足的条件.

\noindent \textbf{1. 欠阻尼$\beta<\omega_0$}\\
此时方程的通解为:
\begin{equation}
  x(t)=\mathrm{e}^{-\beta t}(A\cos{\omega t}+B\sin{\omega t}),\quad \omega\equiv \sqrt{\omega_0^2-\beta^2} \tag{6.11}
\end{equation}
考虑到初值条件,最终解为:
\begin{equation}
  x(t)=\mathrm{e}^{-\beta t}\Big[x_0\cos{\omega t}+\frac{\dot{x_0}+\beta x_0}{\omega}\sin{\omega t}\Big] \tag{6.12}
\end{equation}
由其振荡形式知必存在$x$为负的时刻,故该情形下不满足讨论的条件.

\noindent \textbf{2. 临界阻尼$\beta=\omega_0$}\\
此时方程的通解为:
\begin{equation}
  x(t)=\mathrm{e}^{-\beta t}(A+Bt) \tag{6.13}
\end{equation}
考虑到初值条件,最终解为:
\begin{equation}
  x(t)=x_0\mathrm{e}^{-\beta t}+(\dot{x_0}+\beta x_0)t\mathrm{e}^{-\beta t} \tag{6.14}
\end{equation}
要求其恒大于0,计算结果为:
\begin{equation}
\dot{x_0}\geq-\beta x_0 \tag{6.15}
\end{equation}

\noindent \textbf{3. 过阻尼$\beta>\omega_0$}\\
此时方程的通解为:
\begin{equation}
  x(t)=\mathrm{e}^{-\beta t}(Ae^{\omega' t}+Be^{-\omega' t}), \quad \omega '\equiv \sqrt{\beta^2-\omega_0^2} \tag{6.16}
\end{equation}
考虑到初值条件,最终解为:
\begin{equation}
  x(t)=\displaystyle\frac{1}{2}\mathrm{e}^{-\beta t}\Big[\Big(\frac{\dot{x_0}+(\beta+\omega')x_0}{2\omega'}\Big)e^{\omega' t}+\Big(\frac{-\dot{x_0}+(-\beta+\omega')x_0}{2\omega'}\Big)e^{-\omega' t}\Big] \tag{6.17}
\end{equation}
要求其恒大于0,计算结果为:
\begin{equation}
  \dot{x_0}\geq-(\beta+\omega') x_0 \tag{6.18}
  \end{equation}
$y$方向同理可作上述讨论,最后对$y_0,\dot{y_0}$的要求数学形式与$x$的完全一致.
  
\noindent (3)\\
由对称性知,圆盘的中心始终位于原点,不存在平动速度. 故任意接触点摩擦力的切向分量为:
\begin{equation}
  f_{\tau}=-\frac{1}{3}\mu Mg\frac{l\varOmega}{\sqrt{(l\varOmega)^2+v_0^2}} \tag{6.19}
\end{equation}
因此,转动定理给出:
\begin{equation}
  \displaystyle\frac{1}{2}MR^2\dot{\varOmega}=-\mu Mg\frac{l^2\varOmega}{\sqrt{(l\varOmega)^2+v_0^2}}\tag{6.20}
\end{equation}
积分:
\begin{equation}
  \displaystyle \int_{\varOmega_0}^{\varOmega}\frac{\sqrt{\varOmega^2+(\frac{v_0}{l})^2}}{\varOmega}\mathrm{d}\varOmega=-\frac{2\mu gl}{R^2}\int_0^t\mathrm{d}t \tag{6.21}
\end{equation}
最终结果为:
\begin{equation}
  t=-\displaystyle\frac{R^2}{2\mu gl}\Big\{\sqrt{\varOmega^2+(\frac{v_0}{l})^2}-\sqrt{\varOmega_0^2+(\frac{v_0}{l})^2}-\frac{v_0}{l}\Big[ \ln{\frac{\sqrt{\varOmega^2+(\frac{v_0}{l})^2}+\frac{v_0}{l}}{\sqrt{\varOmega^2+(\frac{v_0}{2l})^2}-\frac{v_0}{l}}}-\ln{\frac{\sqrt{\varOmega_0^2+(\frac{v_0}{2l})^2}+\frac{v_0}{l}}{\sqrt{\varOmega_0^2+(\frac{v_0}{l})^2}-\frac{v_0}{l}}}\Big]\Big\} \tag{6.22}
\end{equation}

\clearpage

\section*{第7题(50分)}
\noindent (1)这个题的出题背景是相对论中的Terrell旋转效应. 我们应同时考虑运动引发的尺缩效应以及运动带来的视觉偏差. 首先,在地面系中,由于尺缩效应,轨道将变为一个以$y$轴为长轴的正椭圆. 其半长轴与半短轴分别为
\begin{equation}
A=R,\quad B=\sqrt{1-\beta^2}R \tag{7.1}
\end{equation}
因此地面系中的椭圆方程可以写为:
\begin{equation}
  \frac{x'^2}{(1-\beta^2)R^2}+\frac{y'^2}{R^2}=1 \tag{7.2}
  \end{equation}
由于光的传播需要时间,观察者位于$y$轴无穷远处,因此$y'$越大的点越提前被看到,假设$y'=0$的点被看到时位于$x'=0$处,我们得到看到时的坐标为:
\begin{equation}
  x=x'+\beta y',\quad y=y' \tag{7.3}
\end{equation}
则看到的曲线方程为
\begin{equation}
  \frac{(x-\beta y)^2}{(1-\beta^2)R^2}+\frac{y^2}{R^2}=1 \tag{7.4}
  \end{equation}
这是一个斜椭圆,我们做$45^\circ$旋转变换,即
\begin{equation}
  \begin{cases}
    \displaystyle
x=\frac{1}{\sqrt{2}}(X-Y)\\
    \displaystyle
y=\frac{1}{\sqrt{2}}(X+Y)
  \end{cases}\tag{7.5}
  \end{equation}
则方程化为:
\begin{equation}
\displaystyle\frac{X^2}{\frac{1+\beta}{1-\beta}}+\frac{Y^2}{\frac{1-\beta}{1+\beta}}=R^2 \tag{7.6}
\end{equation}
我们看到,这恰恰是椭圆的标准方程,因此这个斜椭圆的倾斜角即为$\theta=45^\circ$,半长轴与半短轴分别为:
\begin{equation}
a=\sqrt{\frac{1+\beta}{1-\beta}}R,\quad b=\sqrt{\frac{1-\beta}{1+\beta}}R \tag{7.7}
\end{equation}


\noindent (2)
\noindent \textbf{法一:LRL矢量}\\
这里先简单介绍一下隆格-楞茨(LRL)矢量,它在平方反比力的有心力场问题中很有用.\\

\noindent 当有心力的形式为$\displaystyle \vec{F_r}=-\frac{k}{r^2}\hat{r}$时,LRL矢量定义为:
\begin{equation}
\vec{B}\equiv m\vec{v}\times\vec{L}-km\hat{r} \tag{7.8}
\end{equation}
在本问题中$m$为恒星质量,$L$为恒星相对于黑洞的角动量. 在无相对论修正的情形下,可以证明该矢量是守恒的:
\begin{equation}
  \frac{\mathrm{d}\vec{B}}{\mathrm{d}t}=m\Big(\frac{\mathrm{d}\vec{v}}{\mathrm{d}t}\times \vec{L}+\vec{v} \times \frac{\mathrm{d}\vec{L}}{\mathrm{d}t}-k\frac{\mathrm{d}\hat{r}}{\mathrm{d}t}\Big)=-\frac{mk}{r^2}\hat{r} \times \vec{L}-mk\dot{\theta}\hat{\theta}=0 \tag{7.9}
\end{equation}
因此LRL矢量模长与方向均不变,以下来分别说明. 首先,LRL矢量在极坐标系下可具体写为:
\begin{equation}
  \vec{B}=mL(-v_r\hat{\theta}+v_{\theta}\hat{r})-mk\hat{r}\tag{7.10}
\end{equation}
故可计算其模长:
\begin{equation}
  |\vec{B}|=\sqrt{m^2L^2v_r^2+(mLv_\theta-mk)^2}=\sqrt{m^2L^2v^2+m^2k^2-2m^2v_{\theta}kL}\tag{7.11}
\end{equation}
将$\displaystyle E=\frac{1}{2}m(v_r^2+v_{\theta}^2)-\frac{k}{r},L=mrv_{\theta}$代入,可化简得:
\begin{equation}
  |\vec{B}|=\sqrt{2mEL^2+m^2k^2} \tag{7.12}
\end{equation}
显然此式守恒. 再来考虑方向,我们来计算点乘:
\begin{equation}
  \vec{B}\cdot \vec{r}=Br\cos{\theta}=mL(-v_r\hat{\theta}+v_{\theta}\hat{r})\cdot r\hat{r}-mk\hat{r}\cdot r\hat{r}=L^2-mkr \tag{7.13}
\end{equation}
变形得到:
\begin{equation}
  r=\frac{L^2}{mk+B\cos{\theta}}\sim \frac{p}{1+e\cos{\theta}},\qquad p\equiv\frac{L^2}{mk},e\equiv \frac{B}{mk}\tag{7.14}
\end{equation}
由此证明了恒星的运动轨迹为圆锥曲线(在本题条件下即为椭圆). 当$\theta=0$时,$\displaystyle r=\frac{p}{1+e}$对应“近日点”到长轴端点的距离,此时$\vec{r}$与$\vec{B}$相同,表明\textbf{$\mathrm{LRL}$矢量的方向沿长轴}.

以上讨论的是经典力学情形. 当考虑(狭义)相对论的修正时,角动量和能量仍然守恒,然而LRL矢量不再守恒:
\begin{equation}
  \frac{\mathrm{d}\vec{B}}{\mathrm{d}t}=
  \frac{\mathrm{d}(m\vec{v})}{\mathrm{d}t}\times \vec{L}+m\vec{v}\times \frac{\mathrm{d}\vec{L}}{\mathrm{d}t}-mk\frac{\mathrm{d}\hat{r}}{\mathrm{d}t}-\frac{\mathrm{d}\vec{m}}{\mathrm{d}t}k\hat{r}
  =-\frac{\mathrm{d}m}{\mathrm{d}t}k\hat{r} \tag{7.15}
\end{equation}
此处的$m$是动质量,其值由质能关系给出:
\begin{equation}
  mc^2-\frac{k}{r}=E(\mathrm{const}),\quad \frac{\mathrm{d}m}{\mathrm{d}t}=-\frac{k}{c^2 r^2}\frac{\mathrm{d}r}{\mathrm{d}t} \tag{7.16}
\end{equation}
代入前式,得到:
\begin{equation}
  \frac{\mathrm{d}\vec{B}}{\mathrm{d}t}=\frac{k}{c^2 r^2}\dot{r}\hat{r} \tag{7.17}
\end{equation}
注意到此变化率是沿径向$\hat{r}$的,而进动角正是$\vec{B}$的旋转角,因此只需要考察一个周期内$\vec{B}$的横向变化量:
\begin{equation}
\frac{\text{d}B_\bot}{\text{d}t}=-\frac{k^2}{c^2}\frac{\text{d}(\frac{1}{r})}{\text{d}t}\sin{\theta}\tag{7.18}
\end{equation}
对上式积分,并代入未修正时的轨道方程$\displaystyle\frac{1}{r_0(\theta)}=\frac{1}{p}(1+e\cos{\theta})$,可得:
\begin{equation}
\Delta{B_\bot}=\frac{k^2}{L^2c^2}\int_0^{2\pi}B\sin^2{\theta}\text{d}\theta=\frac{{\pi}k^2B}{c^2L^2}\tag{7.19}
\end{equation}
进动角即为:
\begin{equation}
\Delta{\theta}=\frac{\Delta{B_\bot}}{B}=\frac{{\pi}k^2}{c^2L^2} \tag{7.20}
\end{equation}
本题中$k=GMm$,$L=m\sqrt{GMR}$,故最终可得:
\begin{equation}
\Delta{\theta}=\pi \Big(\frac{GM}{Rc^2}\Big) \tag{7.21}
\end{equation}

\noindent \textbf{法二:Binet方程}\\
在一般的有心力问题中,轨道方程的微分形式由Binet方程给出:
\begin{equation}
\displaystyle \frac{\text{d}^2u}{\text{d}\theta^2}+u=-\frac{m}{L^2u^2}f(\frac{1}{u}),\quad u\equiv \frac{1}{r}\tag{7.22}
\end{equation}
特别地,在平方反比力场中$\displaystyle f(\frac{1}{u})=-\frac{k}{r^2}=-ku^2$,代入整理可得:
\begin{equation}
\displaystyle \frac{\text{d}^2u}{\text{d}\theta^2}+u=+\frac{mk}{L^2} \tag{7.23}
\end{equation}
在无相对论的情形下,由上式可以解得:
\begin{equation}
u_0(\theta)=\frac{1}{r(\theta)}=r_0(1+e\cos\theta)
\tag{7.24}
\end{equation}
其中$A,B$为可由能量$E$与角动量$L$确定的常数(已通过选取适当的极坐标轴使得初相位为0). 当离心率$0<e<1$时,上式便给出常见的椭圆轨道方程.

而当考虑狭义相对论修正时,随黑洞平动的参考系中$m$的值由质能关系$m=\displaystyle\frac{E}{c^2}+\frac{k}{c^2r}$给出,故(7.22)式变为:
\begin{equation}
\displaystyle \frac{\text{d}^2u}{\text{d}\theta^2}+\Big(1-\frac{k^2}{c^2L^2}\Big)u=\frac{Ek}{L^2c^2} \tag{7.25}
\end{equation}
相应地,解应修正为:
\begin{equation}
u(\theta)=r_0\Big(1+e\cos\sqrt{1-\frac{k^2}{c^2L^2}}\theta\Big) \tag{7.26}
\end{equation}
这是一个进动的椭圆,轨道的长轴方向会缓慢旋转. 此时恒星转过“一周”时转过的角度$\varphi$为:
\begin{equation}
\varphi=\displaystyle\frac{2\pi}{\sqrt{1-\frac{k^2}{c^2L^2}}}\sim 2\pi \Big(1+\frac{k^2}{2c^2L^2}\Big)
 \tag{7.27}
\end{equation}
最终得到进动角为:
\begin{equation}
\Delta{\theta}=\varphi-2\pi=\pi\Big(\frac{k}{cL}\Big)^2=\pi\Big(\frac{GM}{Rc^2}\Big)
 \tag{7.28}
\end{equation}
\\
\\

\section*{致谢(碎碎念)}
这张试卷好难!很多小问的计算量或技巧性都达到或超过了复赛水平!但是同学们不要灰心,时间会让一切都变得平常;经过更高难度的训练后,你们可能会发现当年的这些题也不过尔尔,各种套路的背后都有迹可循. 加油!\\

\noindent 感谢以下同学的付出:\\

\noindent 轻松解题的计算之神\quad \textbf{谢航}\\
砍瓜切菜的可耻哥\quad \textbf{谭景仁}\\
热心解决大部分问题的小天才 \quad \textbf{任泓锦}\\
辛苦码字的生产队的驴\quad \textbf{杨泽宇}\\
以及永远亲切关怀着大家的——\textbf{江鸟飞}!


\end{document}

