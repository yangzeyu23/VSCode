\documentclass{ctexart}

\usepackage{yzy}

\title{实验报告}
\class{物理 32}
\name{杨泽宇}
\id{2023011329}

\begin{document}
\maketitle

\section*{摘要}
发现大部分同学不太会写摘要。其实摘要就是一个“小论文”,要用最简洁的话说明文章为啥做,怎么做,做出了啥。语气要客观,不能带感情色彩。例如以下的格式:
本文针对XXX问题,利用XXX方法(仪器、手段),对XXX材料(物质、研究对象)在XXX条件下的XXX行为(性质、响应、特性)进行了测量,测量结果显示XXX(主要结论)。该结果说明了XXX。

\section*{一 实验原理}

%\begin{figure}[h]                                           
%  \includegraphics[scale=0.45]{2.png}
%  \qquad
%  \includegraphics[scale=0.43]{3.png}
%\end{figure}



\section*{二 实验仪器及实验内容}
\subsection*{1.\quad 仪器介绍}

\subsection*{2.\quad 实验步骤}
\subsubsection*{A\quad 测试应力弛豫条件下细线的应力——时间关系}



\subsubsection*{B\quad 数据分析}


\section*{三 数据处理}

\subsection*{A\quad }
\noindent $m_0=79.33\mathrm{g}$\\


%\begin{figure}[htbp]
%  \centering
%  \includegraphics[scale=0.42]{4.png}
%\end{figure}

\subsection*{B\quad 数据处理}
\subsection*{B.1,2}

 
  
\clearpage
\section*{四 讨论}
在本次实验中,我们通过。\\
\noindent \textbf{我们来讨论实验中的以下细节:}\\
\begin{itemize}
  \item 
  \item 
  \item 
\end{itemize}
\textbf{针对以上讨论,我们对实验内容及设计有如下建议:}\\
\noindent 1. 。\\


\section*{五 原始数据}

%\begin{figure}[htbp]
%  \centering
%  \includegraphics[scale=0.4]{1.png}
%\end{figure}

\end{document}

%\begin{figure}[htbp]
% \centering
%  \subfloat[$PG5$]
%  {\includegraphics[width=0.45\textwidth]{PG5.jpg}}
%  \quad
%  \subfloat[$PG8$]
%  {\includegraphics[width=0.45\textwidth]{PG8.jpg}}
%  \caption*{图$11.\quad$的衍射图样}
% \end{figure}


%\begin{figure}[H]
%  \begin{minipage}[c]{0.45\linewidth}
%    \includegraphics[width=\linewidth]{UC7.jpg}
%    \caption*{图$9.(c)\quad \mathrm{UC7}$的衍射图样}
%   \end{minipage}
%    \hspace{1cm}
%    \begin{minipage}[c]{0.8\linewidth}
%    hhh
%    \end{minipage}
%\end{figure}
