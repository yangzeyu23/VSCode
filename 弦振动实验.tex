\documentclass{ctexart}

\usepackage{yzy}

\title{弦振动实验报告}
\class{物理 32}
\name{杨泽宇}
\id{2023011329}

\begin{document}
\maketitle

\section*{摘要}
\noindent 1.利用弦振动方程分析振动在弦上的传播与反射现象,认识边界条件对振动解的影响。\\ 
\noindent 2.观察弦线在正弦信号激励下的受迫振动,认识共振(驻波)现象。\\
\noindent 3.测量不同实验参数下的共振频率,验证共振频率的理论公式。

\section{实验原理}
现有一根半无限长,沿正$x$方向放置的柔软均匀张紧弦线, 其线密度为$\rho$,张力为$T$。对其施加垂直方向的振动激励,满足的波动方程应为:
\begin{equation}
  \displaystyle\frac{\partial^2y}{\partial t^2}-\frac{T}{\rho}\frac{\partial^2y}{\partial x^2}=0
\end{equation}
若记$v=\displaystyle\sqrt{\frac{T}{\rho}}$,则上式的通解为:
\begin{equation}
  y(x,t)=f(x-vt)+g(x+vt)
\end{equation}
其中$y=f(x-vt)$为沿$x$正方向传播的振动,$y=g(x+vt)$为沿$x$负方向传播的振动。可见,振动传播速度仅与弦参数$T$和$\rho$有关。

\indent 当弦线末端固定时,将出现反射现象。此时,弦线的两端均满足固定边界条件,即$y(0,t)=y(L,t)=0$。考虑沿正$x$方向传播的入射波,其波形为$y^+=f(vt+x)$,则沿负$x$方向传播的反射波波形为$y^-=f(vt-x)$。故当入射波传播到末端处时,总波形为:
\begin{equation}
  y(L,t)=y^+(L,t)+y^-(L,t)=0
\end{equation}
此时反射波与入射波大小相同,方向相反,二者完全抵消。此后叠加脉冲将反射,沿$x$负方向传播。

\indent 在周期性正弦激励下,$(1.2)$可以表为$y=A^+e^{\mathrm{i}(\omega t-kx+\varphi^+)}+A^-e^{\mathrm{i}(\omega t+kx+\varphi^-)}$。再将其代入边界条件$y(0,t)=y(L,t)=0$,可得振动的实部解(简便起见,取$\varphi^+=0$):
\begin{equation}
  y(x,t)=-2A^+\sin(kx)\cos(\omega t)
\end{equation}
由此知驻波条件$\sin(kl)=0$,即$L=n\rho/2$,其中$n$为正整数。故驻波时弦上的振动频率为:
\begin{equation}
  f_n=\displaystyle\frac{n}{2L}v=\displaystyle\frac{n}{2L}\displaystyle\sqrt{\frac{T}{\rho}}
\end{equation}
波形中振幅最大的位置为驻波波腹,为0的位置为驻波波节,其间距为$\lambda/2$。$n=1$对应的驻波称为基频,相应驻波称为第一谐波,以此类推。


\section{实验仪器及实验步骤}
\subsection{仪器介绍}
实验室提供6根直径不同的弦。弦线一段由仪器左侧的蝶形螺母固定,另一端通过仪器右侧的尼龙绳线和滑轮组与砝码相连以提供张力。两个位置可以移动的弦码作为弦线的两个固定端,其间距可由固定在基座上上的表尺测量。驱动线圈和信号发生器相连对弦线施加激励,双通道示波器对激励信号和接收线圈介绍到的信号进行测试,以观察振动和共振现象。
\subsection{实验步骤}
\subsubsection*{(1)观察弦的振动,分析$f-n$关系}
\noindent 1.选用粗弦,保持弦长和张力不变,开启信号发生器并调节频率,观察弦振动现象;分别测定$n=1,2,3,4,5,...$时主播的共振频率$f_n$。\\
2.用最小二乘法对$f-n$关系进行直线拟合,计算斜率和不确定度,并与理论值比较。
\subsubsection*{(2)分析$f-L$关系}
\noindent 1.选用中粗弦,保持张力不变,在$30.0-55.0cm$的范围内调节弦长,测量相应的基频$f$。\\
2.用最小二乘法对$f_1-\frac{1}{L}$关系进行直线拟合,计算斜率和不确定度,并与理论值比较。
\subsubsection*{(3)分析$f-T$关系}
\noindent 1.选用细弦,保持弦长不变,在$1.96N-11.76N$的范围内调节张力,测量相应的基频$f_1$。\\
2.用最小二乘法对$\ln{f_1}-\ln{T}$关系进行直线拟合,计算斜率和不确定度,并与理论值比较。
\subsubsection*{(4)分析$f-\rho$关系}
\noindent 1.保持$n,L,T$不变,分别测量剩余三根弦的基频频率$f$。\\
2.用最小二乘法对$\ln{f}-\ln{\rho}$关系进行直线拟合,计算斜率和不确定度,并与理论值比较。
\subsubsection*{(5)分析弦参数与基频及波速的关系}
\noindent 选取$(1)-(4)$的实验数据,计算弦上振动波速的实验值$v=\displaystyle\frac{2L}{n}f$和理论值$v=\displaystyle\sqrt{\frac{T}{\rho}}$,比较其相对偏差。

\section{数据处理}
\clearpage
\subsection*{(1)分析$f-n$关系}
\noindent 实验条件:6号弦,$\rho=0.00936\mathrm{kg/m},\quad L=50.0\mathrm{cm},\quad T=9.80\mathrm{N}$.\\

\begin{table}[h]
  \caption{$f-n$关系测定} \vspace{0.7em}
  \centering
  \begin{tabular}{cccccc}
    \hline
    n& 1& 2& 3& 4& 5\\
    $f/\mathrm{Hz}$& 31.93 & 64.66 & 97.23 & 137.17 & 165.85 \\
    \hline
    \end{tabular}
\end{table}

\noindent 最小二乘的结果为:$b_1=32.852
,\quad s_{b_1}=0.169, \quad U_{b_1}=0.469$. \\
故斜率测定值为$b_1=32.9\pm0.5$,理论值为$b_1*=\frac{1}{2L}\sqrt{\frac{T}{\rho}}=32.4$,实验值较理论值略大.\\

\subsection*{(2)分析$f-L$关系}
\noindent 实验条件:4号弦,$\rho=0.00350\mathrm{kg/m},\quad T=9.80\mathrm{N},\quad n=1$.\\
\begin{table}[h]
  \caption{$f-L$关系测定} \vspace{0.7em}
  \centering
  \begin{tabular}{ccccccc}
    \hline
    $L/\mathrm{cm}$& 30.0& 35.0& 40.0& 45.0& 50.0& 55.0\\
    $f/\mathrm{Hz}$& 84.45 & 72.00 & 63.51 & 57.80 & 51.82 & 46.99\\
    \hline
    \end{tabular}
\end{table}

\noindent 最小二乘的结果为:$b_1=25.510
,\quad s_{b_1}=0.135, \quad U_{b_1}=0.346$. \\
故斜率测定值为$b_1=25.5\pm0.3$,理论值为$b_1*=\frac{1}{2}\sqrt{\frac{T}{\rho}}=26.5$,实验值较理论值略小.\\

\subsection*{(3)分析$f-T$关系}
\noindent 实验条件:1号弦,$\rho=0.00055\mathrm{kg/m},\quad L=50.0\mathrm{cm},\quad n=1$.\\
\begin{table}[h]
  \caption{$f-T$关系测定} \vspace{0.7em}
  \centering
  \begin{tabular}{ccccccc}
    \hline
    $T/\mathrm{N}$& 1.96 & 3.92& 5.88& 7.84& 9.80& 11.76\\
    $f/\mathrm{Hz}$& 60.19& 85.11 & 104.51& 120.70  & 130.90 &143.16\\
    \hline
    \end{tabular}
\end{table}

\noindent 最小二乘的结果为:$b_1=0.485
,\quad s_{b_1}=0.009, \quad U_{b_1}=0.024$. \\
故斜率测定值为$b_1=0.49\pm0.02$,理论值为$b_1*=0.5$,实验值较理论值略小.\\

\subsection*{(4)分析$f-\rho$关系}
\noindent 实验条件:$L=50.0\mathrm{cm},\quad T=9.80\mathrm{N},\quad n=1$.\\
\begin{table}[!htbp]
  \centering
  \caption{$f-\rho$关系测定}\vspace{0.7em} \label{tab:aStrangeTable}%添加标题 设置标签
  \begin{tabular}{ccc}
  \toprule
  弦序号& $\rho/\mathrm{kg\cdot m^{-1}}$& $f/\mathrm{Hz}$ \\
  \midrule
  1& 0.00055& 130.90  \\
  2& 0.00098& 101.32  \\
  3& 0.00191& 67.25  \\
  4& 0.00350& 51.82  \\
  5& 0.00578& 37.75  \\
  6& 0.00936& 31.93  \\
  \bottomrule
  \end{tabular}
  \end{table}

  \noindent 最小二乘的结果为:$b_1=-0.47
  ,\quad s_{b_1}=0.06, \quad U_{b_1}=0.18$. \\
故斜率测定值为$b_1=-0.5\pm0.2$,理论值为$b_1*=-0.5$,实验值较与论值基本吻合.\\

\subsection*{(5)分析弦参数与基频及波速的关系}
我们以基频为例,计算在第一激发态下不同弦线上振动传播速度的实验值$v=2Lf$和理论值$v=\sqrt{\frac{T}{\rho}}$,并比较其相对偏差. 结果如下:\\
\begin{table}[h]
    \caption{波速$v$的实验与理论值计算} \vspace{0.7em}
    \centering
    \begin{tabular}{ccccccc}
      \hline
      绳序号& 1& 2& 3& 4& 5& 6\\
      实验值/$\mathrm{m\cdot s^{-1}}$& 130.90& 101.32& 67.25& 51.82& 37.75& 31.93\\
      理论值/$\mathrm{m\cdot s^{-1}}$& 133.48 & 100.00 & 71.63 & 52.92 & 41.18 & 32.36\\
      相对偏差& -1.9\% & +1.3\% & -6.1\% & -2.1\% & -8.3\% & -1.3\% \\
      \hline
      \end{tabular}
  \end{table}   

6组中的相对偏差均在$\pm10\%$以内,可见波速的实验测定值与理论预测值基本吻合.

\section{讨论}

\noindent \textbf{1.(实验内容6)}  首先,探测线圈应尽可能放置在弦驻波的波腹处,这样可以放大观测到的振幅值,更容易确定共振现象的发生。其次,探测线圈和接收线圈之间的距离应当大于$10\mathrm{cm}$,以防二者的电磁信号相互干扰,对观测产生影响.\\
\noindent \textbf{2.(实验内容7)}  可以采用的方法有:(1) 尽可能地将探测线圈放置在弦驻波的波腹处,这样可以放大观测到的振幅值,更容易确定共振现象的出现. (2) 提前由实验条件计算出频率的理论值,并在其附近调节信号发生器的频率,可以有效缩小调节的范围.(3)在可能的共振频率附近作细致调节,且调节后适当等待以使弦振动进入稳态,这有助于更精确地确定共振发生的频率值.\\

\section{原始数据}

\begin{figure}  
  \centering                                          
  \includegraphics[scale=0.35]{1.jpg}
  \end{figure}


\end{document}