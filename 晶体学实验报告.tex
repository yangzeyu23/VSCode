\documentclass{ctexart}

\usepackage{yzy}

\title{晶体学实验报告}
\class{物理 32}
\name{杨泽宇}
\id{2023011329}

\begin{document}
\maketitle

\section*{摘要}
本实验从一维光栅出发,寻找衍射图样与晶体结构之间的关系,最后测定未知晶体的结构。通过对数据的处理分析,我们将确定未知晶体的晶格矢量、晶胞的对称性和晶胞结构等细节,加深对晶体学的认识。


\section*{一\quad 实验原理}
\subsection*{引言}
晶体是由基本单元晶胞在空间周期性排列形成的结构。设$\rho(\bm{r})$为描述晶体结构的实函数,$\bm{a_1},\bm{a_2},\bm{a_3}$为一组线性无关的晶格矢量,则晶体结构符合下式:
\begin{equation}
  \rho(\bm{r}+e\cdot \bm{a_1}+f\cdot \bm{a_2}+g\cdot \bm{a_3})=\rho(\bm{r}),\quad e,f,g\in \mathbb{Z}
\end{equation}

晶体具有周期性,故可以通过衍射确定其结构。考虑一强度为$I_0$,波长为$\lambda$,波矢为$\bm{k_i}$的光垂直入射晶体样品,若发生的散射是完全弹性的,则散射角$\theta$处的衍射波的波矢$\bm{k_s}$模长与$\bm{k_i}$相等:
\begin{equation}
  |\bm{k_s}|=|\bm{k_i}|,\quad q\equiv |\bm{k_s}-\bm{k_i}|=2k_i\sin\theta/2
\end{equation}
这里的$\bm{q}$称为\textbf{散射矢量}。在本实验中总可以认为$\theta \ll 1$,且$q \ll k_i$,故有近似:
\begin{equation}
  q=2k_i\sin\theta/2\approx k_i\theta
\end{equation}
从而,$\bm{q}$将给出衍射光束的复振幅\textbf{结构因子}$F(\bm{q})=|F|e^{i\varphi}$,而衍射光强$I(\bm{q})=|F(\bm{q})|^2=FF^*$。对测试得到的衍射光强进行分析,即可得到晶体结构$\rho(\bm{r})$。
\subsection*{A\quad 一维晶体}
考虑光栅常量为$a$,狭缝宽为$b$的衍射光栅,其构成一类最简单的一维晶体。当观察其夫琅禾费衍射时,光强与衍射角$\theta$的关系为:
\begin{equation}
  I(\theta)=\frac{I_0}{N^2}\left(\frac{\sin(N\pi a\sin\theta/\lambda)}{\sin{(\pi a\sin\theta/\lambda)}}\right)^2\left(\frac{\sin(\pi b\sin\theta/\lambda)}{\pi b\sin\theta/\lambda}\right)^2
\end{equation}
其中$I_0$为$\theta=0$时的光强,$N$为光栅的刻痕数。利用散射矢量$q$对上式作代换,得:
\begin{equation}
  I(q)=\frac{I_0}{N^2}\left(\frac{\sin Nqa/2}{\sin{qa/2}}\right)^2\left(\frac{\sin{qb/2}}{qb/2}\right)^2
\end{equation}
由此可以确定,光栅常量为$a$的光栅的第$h$个衍射极大对应的散射矢量为:
\begin{equation}
  q=\frac{2\pi }{a}h=q_1h,\quad h\in \mathbb{Z} 
\end{equation}
\subsection*{B\quad 二维晶体}
考虑格矢为$\bm{a_1},\bm{a_2}$(其夹角为$\alpha\leqslant 90°$)的二维晶体。不难证明,总可以在其平面内作一系列平行线使得其通过所有格点,从而该二维晶体的衍射可看作是光栅常量为平行线线距的光栅衍射。

设$\bm{q_1},\bm{q_2}$为对应于$\bm{a_1},\bm{a_2}$的散射矢量,则$\bm{q_1},\bm{q_2}$分别与$\bm{a_1}$或$\bm{a_2}$之一垂直。不妨取$\bm{q_1}\perp\bm{a_2}$,则有:
\begin{equation}
  |\bm{q_1}|=\frac{2\pi}{a_1sin\alpha},\quad |\bm{q_2}|=\frac{2\pi}{a_2sin\alpha}
\end{equation}
\subsection*{C\quad 晶体的对称性}
晶胞的对称性会引起衍射图样的对称和缺级。衍射斑点强度分布的典型对称性包括镜像对称,和绕某轴旋转$(360°\cdot n/m, n\in \mathbb{Z})$时能使衍射图样复原的$m$阶旋转对称$C_m$。

具体到二维晶体,其旋转对称性$C_n$与平移对称性一起给出如下关系:
\begin{equation}
  \cos{\theta}=\frac{1-n}{2}
\end{equation}
上式反映了平移与旋转的联系,其中$n$为整数,$\theta$为可使图样复原(即回到等价位形)的旋转角。由此可以确定二维晶体可能具有的旋转对称性:
\begin{figure}[htbp]
  \centering
  \includegraphics[scale=0.15]{1.jpg}
\end{figure}
\subsection*{D\quad 晶体结构测定}
入射光经晶体样品散射后衍射图样的复振幅(即结构因子)由傅里叶变换给出:
\begin{equation}
  F(\bm{q})\sim \int \rho(\bm{r})e^{i\bm{q}\cdot \bm{r}}d\bm{r}
\end{equation}
相应地,傅里叶逆变换给出晶胞的结构。对于离散的衍射斑点,积分可以化成求和:
\begin{equation}
  \rho(\bm{r})\sim \sum|F(\bm{q})|e^{i\varphi} e^{-i\bm{q}\cdot \bm{r}}
\end{equation}
事实上,上式只需对最亮的部分衍射斑求和即可。然而,实验中测得$I(\bm{q})$并得到$F(\bm{q})$以后无法直接确定相位$\varphi$,故无法直接解出$\rho(\bm{r})$。解决该问题的方法通常是先取近似的初始相位,利用(1.10)式得到$\rho(\bm{r})$,再将得到的$\rho(\bm{r})$代入(1.9)式计算新的相位,不断重复直到收敛。如果已知某晶体的结构与待测晶体相似,则可使用该晶体的相位作为初始相位。


\section*{二\quad 实验仪器及实验内容}
\subsection*{1.\quad 仪器介绍}
实验主要仪器包括激光笔(波长为650nm)、镀有样品图案的铬版、干板夹(用于固定铬版)、激光功率计、坐标值(用作光屏)、台灯、量角器及直尺等。实验中当注意激光安全,同时小心保存铬版,并避免触摸其镀有图案的表面。
\subsection*{2.\quad 实验步骤}
\subsubsection*{A\quad 一维晶体}
\noindent 按照图1所示光路,观察并拍摄样品DG1$\sim$DG5的衍射图样,记录数据并计算每个样品的$q_1$和$a$。
\subsubsection*{B\quad 二维晶体}
\noindent 1.观察并拍摄样品UC1$\sim$UC4的衍射图样,记录数据并计算$a_{uc1},a_{uc2},a_{uc3},a_{uc4}$。\\
\noindent 2.确定并分析样品UC1$\sim$UC4与图2中各个晶胞的对应关系。\\
\noindent 3.观察并拍摄样品UC5$\sim$UC7(简单二维晶体)的衍射图样,确定各个样品的参量$a_1,a_2$及其夹角$\alpha$。
\subsubsection*{C\quad 晶体的对称性}
\noindent 1.说明图3所示晶胞形成的晶体具有的对称性(镜像对称轴在图上标出,旋转对称性在下方注明)。\\
\noindent 2.观察并拍摄样品PG1,2,5,8的衍射图样,确定其具有的对称性,并其分析与图3中晶胞的对应关系。\\
\noindent 3.观察并拍摄样品UC8的衍射图样,并判断其是否为晶体。
\subsubsection*{D\quad 晶体结构测定}
\noindent 观察并拍摄样品MR1的衍射图样,确定其晶胞是图4中的哪一个。
\begin{figure}[h]                                           
  \includegraphics[scale=0.25]{1.png}
  \qquad
  \includegraphics[scale=0.29]{2.png}
\end{figure}
\begin{figure}[h]                                           
  \includegraphics[scale=0.32]{3.png}
  \qquad
  \includegraphics[scale=0.32]{4.png}
\end{figure}

注:MR0和MR1的晶胞都由$4\times 4$的透光($\rho=1$)或不透光($\rho=0$)的方格组成,MR0的晶胞结构已知,如图5所示。MR1的晶胞结构为图4中的一个,且与MR0的晶胞结构相似。图6给出了MR0晶体$|h|,|k|\leqslant 2$
的衍射斑点的相位。
\begin{figure}[h]                                           
  \includegraphics[scale=0.36]{5.png}
  \qquad
  \includegraphics[scale=0.36]{6.png}
\end{figure}

\section*{三\quad 数据处理}
\subsection*{A\quad 一维晶体}
\noindent \textbf{光栅至屏的垂直距离为:$L=80.00cm$}
\begin{figure}[H]
  \begin{minipage}[c]{0.4\linewidth}
    \includegraphics[width=\linewidth]{DG1.jpg}
    \caption*{图$7.(a)\quad \mathrm{DG1}$的衍射图样}
    \end{minipage}
    \hspace{1cm}
    \begin{minipage}[c]{0.6\linewidth}
    \noindent $s_1=2.62cm$\\
    故:\\
    $q_1\approx k\theta_1 \approx \frac{2\pi}{\lambda}\frac{s_1}{L}=\frac{2\pi}{650nm} \times \frac{2.62cm}{80.00cm}=3.166\times 10^5m^{-1}$\\
    $a=\frac{2\pi}{q_1}=1.985\times 10^{-5}m$
    \end{minipage}
\end{figure}
\begin{figure}[H]
  \begin{minipage}[c]{0.4\linewidth}
    \includegraphics[width=\linewidth]{DG2.jpg}
    \caption*{图$7.(b)\quad \mathrm{DG2}$的衍射图样}
    \end{minipage}
    \hspace{1cm}
    \begin{minipage}[c]{0.6\linewidth}
      \noindent $s_1=1.05cm$\\
      故:\\
      $q_1\approx k\theta_1 \approx \frac{2\pi}{\lambda}\frac{s_1}{L}=\frac{2\pi}{650nm} \times \frac{1.05cm}{80.00cm}=1.269\times 10^5m^{-1}$\\
      $a=\frac{2\pi}{q_1}=4.952\times 10^{-5}m$
    \end{minipage}
\end{figure}
\begin{figure}[H]
  \begin{minipage}[c]{0.4\linewidth}
    \includegraphics[width=\linewidth]{DG3.jpg}
    \caption*{图$7.(c)\quad \mathrm{DG3}$的衍射图样}
    \end{minipage}
    \hspace{1cm}
    \begin{minipage}[c]{0.6\linewidth}
      \noindent $s_1=0.65cm$\\
      故:\\
      $q_1\approx k\theta_1 \approx \frac{2\pi}{\lambda}\frac{s_1}{L}=\frac{2\pi}{650nm} \times \frac{1.05cm}{80.00cm}=7.854\times 10^4m^{-1}$\\
      $a=\frac{2\pi}{q_1}=8.000\times 10^{-5}m$
    \end{minipage}
\end{figure}
\begin{figure}[H]
  \begin{minipage}[c]{0.4\linewidth}
    \includegraphics[width=\linewidth]{DG4.jpg}
    \caption*{图$7.(d)\quad \mathrm{DG4}$的衍射图样}
    \end{minipage}
    \hspace{1cm}
    \begin{minipage}[c]{0.6\linewidth}
      \noindent $s_1=0.65cm$\\
      故:\\
      $q_1\approx k\theta_1 \approx \frac{2\pi}{\lambda}\frac{s_1}{L}=\frac{2\pi}{650nm} \times \frac{1.05cm}{80.00cm}=7.854\times 10^4m^{-1}$\\
      $a=\frac{2\pi}{q_1}=8.000\times 10^{-5}m$
    \end{minipage}
\end{figure}
\begin{figure}[H]
  \begin{minipage}[c]{0.4\linewidth}
    \includegraphics[width=\linewidth]{DG5.jpg}
    \caption*{图$7.(e)\quad \mathrm{DG5}$的衍射图样}
    \end{minipage}
    \hspace{1cm}
    \begin{minipage}[c]{0.6\linewidth}
      \noindent $s_1=0.65cm$\\
      故:\\
      $q_1\approx k\theta_1 \approx \frac{2\pi}{\lambda}\frac{s_1}{L}=\frac{2\pi}{650nm} \times \frac{1.05cm}{80.00cm}=7.854\times 10^4m^{-1}$\\
      $a=\frac{2\pi}{q_1}=8.000\times 10^{-5}m$
    \end{minipage}
\end{figure}    


\subsection*{B\quad 二维晶体}
\subsubsection*{B.1,2\quad 样品UC1$\sim$UC4}
\noindent \textbf{光栅至屏的垂直距离为:$L=80.00cm$}
\begin{figure}[htbp]
  \begin{minipage}[c]{0.43\linewidth}
    \includegraphics[width=\linewidth]{UC1.jpg}
    \caption*{图$8.(a)\quad \mathrm{UC1}$的衍射图样}
    \end{minipage}
    \hspace{1.2cm}
    \begin{minipage}[c]{0.6\linewidth}
      \textbf{缺级:3的整数倍数级(零级极大除外)}\\
      \noindent $s_1=1.75cm$\\
      故:\\
      $a=\lambda\frac{L}{s_1}=650nm\times \frac{80cm}{1.75cm}=2.971\times 10^{-5}m$
    \end{minipage}
\end{figure}
\begin{figure}[htbp]
  \begin{minipage}[c]{0.43\linewidth}
    \includegraphics[width=\linewidth]{UC2.jpg}
    \caption*{图$8.(b)\quad \mathrm{UC2}$的衍射图样}
    \end{minipage}
    \hspace{1.2cm}
    \begin{minipage}[c]{0.6\linewidth}
      \noindent \textbf{缺级:偶数级(零级极大除外)}\\
      $s_1=2.63cm$\\
      故:\\
      $a=\lambda\frac{L}{s_1}=650nm\times \frac{80cm}{2.63cm}=1.977\times 10^{-5}m$\\
    \end{minipage}
\end{figure}
\begin{figure}[htbp]
  \begin{minipage}[c]{0.43\linewidth}
    \includegraphics[width=\linewidth]{UC3.jpg}
    \caption*{图$8.(c)\quad \mathrm{UC3}$的衍射图样}
    \end{minipage}
    \hspace{1.2cm}
    \begin{minipage}[c]{0.6\linewidth}
      \noindent \textbf{缺级:$h+k$为奇数的级次($h,k$为一维衍射级)}\\
      $s_1=2.59cm$\\
      故:\\
      $a=\lambda\frac{L}{s_1}=650nm\times \frac{80cm}{2.59cm}=2.008\times 10^{-5}m$
    \end{minipage}
\end{figure}
\begin{figure}[H]
  \begin{minipage}[c]{0.43\linewidth}
    \includegraphics[width=\linewidth]{UC4.jpg}
    \caption*{图$8.(d)\quad \mathrm{UC4}$的衍射图样}
    \end{minipage}
    \hspace{1.2cm}
    \begin{minipage}[c]{0.6\linewidth}
      \noindent \textbf{缺级:3的整数倍数级(零级极大除外)}\\
     $s_1=1.75cm$\\
      故:\\
      $a=\lambda\frac{L}{s_1}=650nm\times \frac{80cm}{1.75cm}=2.971\times 10^{-5}m$
    \end{minipage}
\end{figure}
由$UC3$特殊的缺级情况可知,其对应的晶胞是图2中具有平移性结构的D。而由图2中$a_A<a_B$的条件知,A对应的是$UC2$或$UC3$此类光栅常数$a\approx 20\mu m$的光栅,从而可推出$UC2$对应的是图2中的A。因此B和C对应的是$UC1$及$UC4$,而其光栅常数$a\approx 30\mu m$相近,从而由图2知C的透光区域更小,其零级主极大斑点应当暗于B。观察图片知UC1的零级斑点更强,故$UC1$对应的是图2中的B,$UC4$对应的是图2中的C。\\

\textbf{综上分析,UC1$\sim $UC4对应图2中晶胞的序号依次应为:B,A,D,C。}

\clearpage
\subsubsection*{B.3\quad 样品UC5$\sim$UC7}
\noindent \textbf{光栅至屏的垂直距离为:$L=80.00cm$}
\begin{figure}[H]
  \begin{minipage}[c]{0.45\linewidth}
    \includegraphics[width=\linewidth]{UC5.jpg}
    \caption*{图$9.(a)\quad \mathrm{UC5}$的衍射图样}
    \end{minipage}
    \hspace{1cm}
    \begin{minipage}[c]{0.8\linewidth}
    $s_{x1}=5.25cm/2=2.625cm$\\
    故:$\quad a_1=\lambda\frac{L}{s_{x1}}=650nm\times \frac{80cm}{2.625cm}=1.981\times 10^{-5}m$\\
    \noindent \\
    $s_{x2}=7.90cm/6=1.32cm$\\
    故:$\quad a_1=\lambda\frac{L}{s_{x1}}=650nm\times \frac{80cm}{1.32cm}=3.949\times 10^{-5}m$\\
    \noindent \\
    显然x与y两轴正交,故$\alpha=90°$。
    \end{minipage}
\end{figure}
\begin{figure}[H]
  \begin{minipage}[c]{0.45\linewidth}
    \includegraphics[width=\linewidth]{UC6.jpg}
    \caption*{图$9.(b)\quad \mathrm{UC6}$的衍射图样}
    \end{minipage}
    \hspace{1cm}
    \begin{minipage}[c]{0.8\linewidth}
    $s_{x1}=\sqrt{(7+0.27+0.58)^2+(4+0.45+0.75)^2}cm/4=2.35cm$\\
    故:$\quad a_1=\lambda\frac{L}{s_{x1}}=650nm\times \frac{80cm}{2.35cm}=2.21\times 10^{-5}m$\\
    \noindent \\
    $s_{x2}=\sqrt{(7+0.08+0.80)^2+(3+0.27+0.6)^2}cm/6$=1.46cm\\
    故:$\quad a_1=\lambda\frac{L}{s_{x1}}=650nm\times \frac{80cm}{1.46cm}=3.56\times 10^{-5}m$\\
    \noindent \\
    $\alpha=\arccos\Big|\frac{(2.35\times 4)^2+(1.46\times 6)^2-12.11^2}{2\times (2.35\times 4) \times (1.46 \times 6) }\Big|=83.1°$
    \end{minipage}
\end{figure}
\begin{figure}[H]
  \begin{minipage}[c]{0.45\linewidth}
    \includegraphics[width=\linewidth]{UC7.jpg}3

    \caption*{图$9.(c)\quad \mathrm{UC7}$的衍射图样}
    \end{minipage}
    \hspace{1cm}
    \begin{minipage}[c]{0.8\linewidth}
    $s_{x1}=\sqrt{(6+0.05+0.85)^2+5.1^2}cm/6$=1.43cm\\
    故:$\quad a_1=\lambda\frac{L}{s_{x1}}=650nm\times \frac{80cm}{1.43cm}=3.63\times 10^{-5}m$\\
    \noindent \\
    $s_{x2}=\sqrt{6.80^2+(5+0.18+0.15)^2}cm/6$=1.44cm\\
    故:$\quad a_1=\lambda\frac{L}{s_{x1}}=650nm\times \frac{80cm}{1.44cm}=3.61\times 10^{-5}m$\\
    \noindent \\
    $\alpha=\arccos\Big|\frac{(1.43\times 6)^2+(1.44\times 6)^2-(6.45/4)^2}{2\times (1.43\times 6) \times (1.44 \times 6) }\Big|=68.4°$
    \end{minipage}
\end{figure}
\subsection*{C\quad 晶体的对称性}
\subsubsection*{C.1}
\noindent \textbf{分析:}晶胞K不具有轴对称性,而旋转360°,90°,180°均可使其复原,故具有的旋转对称性为$C_1,C_2,C_4$。晶胞L关于水平轴镜像对称,仅旋转360°可以复原,故具有的旋转对称性为$C_1$。晶胞M由结构单元平移而成,仅旋转360°可以复原,故具有的对称性为$C_1$。晶胞N关于两个对角线轴线镜像对称,且旋转360°,180°可复原,故具有的旋转对称性为$C_1,C_2$。具体标注如下:\\
\begin{figure}[H]
  \centering
  \includegraphics[scale=0.24]{11.png}
  \caption*{图$10.$ 对称性与对称轴示意图}
\end{figure}
\clearpage
\subsubsection*{C.2\quad 样品PG1,2,5,8}
\begin{figure}[htbp]
  \centering
  \subfloat[$PG1$]
  {\includegraphics[width=0.45\textwidth]{PG1.jpg}}
  \quad    
  \subfloat[$PG2$]
  {\includegraphics[width=0.45\textwidth]{PG2.jpg}}
  \quad
  \subfloat[$PG5$]
  {\includegraphics[width=0.45\textwidth]{PG5.jpg}}
  \quad
  \subfloat[$PG8$]
  {\includegraphics[width=0.45\textwidth]{PG8.jpg}}
  \caption*{图$11.\quad \mathrm{PG1,2,5,8}$的衍射图样}
\end{figure}
\noindent \textbf{衍射图样的对称性分析:}\\
\noindent 1.样品PG1的衍射图样关于零级主极大所处的水平轴镜像对称;旋转对称性不明显,但至少具有旋转对称性$C_1$。\\
\noindent 2.样品PG2的衍射图样比较杂乱,无明显特殊对称性,但至少具有旋转对称性$C_1$。\\
\noindent 3.样品PG5的衍射图样关于过零级主极大的两正交对角轴线镜像对称,且观察知具有旋转对称性$C_1,C_2$。\\
\noindent 4.样品PG8的衍射图样无明显镜像对称性,但观察知具有旋转对称性$C_1,C_2,C_4$。\\
\textbf{综上分析,PG1,2,5,8对应图3中晶胞的序号依次为:L,M,N,K。}
\subsubsection*{C.3\quad 样品UC8}
\begin{figure}[H]
  \centering
  \includegraphics[scale=0.08]{UC8.jpg}
  \caption*{图$12.\quad \mathrm{UC8}$的衍射图样}
\end{figure}
观察分析可知,样品UC8的衍射图样具有明显的旋转对称性$C_5$,然而这是二维晶体不可能具有的对称性。因此该样品不可能是晶体。
\clearpage
\subsection*{D\quad 晶体结构测定}
\begin{figure}[H]
  \centering 
  \includegraphics[scale=0.08]{MR1.jpg}
  \caption*{图$13.\quad \mathrm{MR1}$的衍射图样}
\end{figure}
晶体各衍射斑点的相对光强与相应的晶胞振幅透射率计算如下:
\noindent \\

\begin{figure}[htbp]
  \centering
  \includegraphics[scale=0.55]{00.png}
  \caption*{表2. 晶胞振幅透射率的计算}
\end{figure}
\noindent \textbf{晶胞结构分析:}\\
考察各级衍射斑点的透射率。半定量地,不妨设立一个数字作为透射率大小的“阈值”,并认为透射率高于该阈值的的斑点对应透光的方格,否则对应不透光的方格。观察可知,对阈值一个合适的选择是30,则透光的方格集为$\{(\chi ,\gamma )|(0,0),(1,1),(1,2),(2,1),(2,3),(3,2)\}$,这与图4中晶胞X的结构相对应。\\
\clearpage
\section*{四\quad 讨论}

在本次实验中,我们通过观察晶体样品的衍射图样,计算并分析了晶体的晶格矢量与对称性,并利用相位法尝试确定了晶体的完整结构。我们来讨论实验中的以下细节:
\begin{itemize}
  \item \textbf{缺级问题.}\quad 谱线的缺级现象是由晶胞大小与晶格矢量模长的比值决定的;当该比值取得整数时,对应级次的衍射斑点将会消失。然而在样品UC3的衍射图样中,出现了$h+k$为奇数级次时的特殊缺级现象,从而表明晶胞内部的平移性结构将会改变晶体的结构因子,进而影响衍射光强的分布函数。
  \item \textbf{二维晶体的晶格矢量.} 在实际测量中可以意识到,$\bm{a_1}$与$\bm{a_2}$的选择并不是唯一的,即有两种形状不同但地位上等价的平行四边形的晶胞可以被认为是晶体的基本结构单元。相应地,$\bm{a_1}$与$\bm{a_2}$之间的夹角$\alpha$取值也有两种可能。而IPHO试题答案中考虑的“晶胞三角形”,则完整地给出了这两种不同的“基矢选择”蕴含的信息:事实上,正是相邻的“晶胞三角形”的不同组合给出了构成平行四边形晶胞的两种不同方式。
  \item \textbf{晶体结构测定.} 在处理数据时,我们发现晶胞的振幅透射率测量结果出现了负值,物理上不合法。这是因为我们计算时只迭代了一次,尚未收敛到最终的合理解。不过由于选取了合适的初始相位(即样品MR0相位值)与阈值,我们仍然能够较为准确地确定晶胞结构。
\end{itemize}
\textbf{针对以上讨论,我们认为实验内容及设计可有如下改进:}\\
\noindent 1. 在分析衍射图样的同时,可以尝试测定不同级次衍射斑点的光强,以定量地分析晶体的结构因子,获取相位信息;同时也可以验证巴比涅定理。\\
\noindent 2. 在晶体结构测定中,可以考虑多次迭代计算晶胞的振幅透射率,以获得更准确的结果。\\
\noindent 3. 可以设计实验,探讨样品$UC8$的微观结构,或分析其具有$C_5$对称性的原因。\\
\noindent 4. 可以考虑探究三维晶体的衍射图样以及各类对称性。(可能难度很大,可以考虑选取较为简单的晶胞结构或者设计简化实验,如探究平行的双层二维晶体的衍射问题)\\

\end{document}
